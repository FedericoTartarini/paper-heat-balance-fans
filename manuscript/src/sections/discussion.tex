%! Author = sbbfti
%! Date = 22/10/2020

\section{Discussion}\label{sec:discussion}

Electric fans are a cheaper and more energy-efficient alternative to compressor-based air conditioning.
Fans require less energy to operate, have lower operational and maintenance costs, and do not use refrigerants that potentially harm the environment.
However, several public health guidelines discourage their use when \acf{t-db} exceed 35~$^{\circ}$C~\cite{WMO2015}.
% note - SS you need to be 100 sure about this point. Some of them do not discourage them, they state that they may not be sufficient to avoid heat-related illness
Our results show that this recommendation is too conservative and that the use of fans at higher ambient temperatures could yield substantial benefits.
While sensible heat gain increases as the \ac{t-db} exceeds \acf{t-sk}, this is more than offset by the increase in latent heat loss by elevated air speed. 
This is true for more than 99~\% of recorded extreme heat events worldwide.
For example, increasing the average air speed from 0.2 m/s to 0.8 m/s raises the temperature above which heat strain would occur by an average of 1.7 (1.4, 2.1)~$^{\circ}$C [mean, (Q1, Q3)].
In very hot and dry conditions (e.g., \ac{t-db}~=~47~$^{\circ}$C, \ac{rh}~=~15~\%) fans should only be used in combination with evaporative cooling technologies to prevent heat stress from occurring.
On the other hand we found that people should not use electric fans (e.g., \ac{v} higher than 0.8~m/s) when indoor conditions exceed \ac{t-op}~=~50~$^{\circ}$C and \ac{rh}~=~40~\%, \ac{t-op}~=~40~$^{\circ}$C and \ac{rh}~=~82~\% or \ac{t-op}~=~37.5~$^{\circ}$C and \ac{rh}~=~100~\%.
% todo CH the above sentence is a little hard to interpret. You are contrasting it ("on the other hand") to hot/dry conditions, yet you list three conditions that are not easy to characterize. What about just referring to the graph?
It should be noted that such extreme weather events are rare and may only occur in few locations across the world~\cite{ASHRA2017}.
We also compared these data with the records available in the Emergency Events Database (EM-DAT)~\cite{EMDATThe70:online}.
% todo CH paragraph below would be great if we could provide something a little more specific here. For example, could we do an analysis were we look at the number of deaths at each temperature bin, estimate the effective temperature decrease of 0.8 m/s air speed, and then estimate the number (or percentage_ of avoided deaths
A total of 122 heatwave events in the EM-DAT had information on the maximum air temperature recorded.
These heatwaves were the cause of approximately 117000 deaths, out of which a total of 102876 and 3803 people died during heatwaves with maximum temperatures lower than 45~$^{\circ}$C and 40~$^{\circ}$C, respectively.
During heatwaves as \ac{t-db} increases \ac{rh} generally decreases, hence, it may be hypothesized that the use of electric fans would have been beneficial in most of those scenarios.

Our results validate and extend the findings of \mycite{Jay2015} that elevated air movement is beneficial even when \ac{t-db} exceeds \acf{t-sk}.
This is significant because cooling the air to avoid heat stress is both environmentally and economically disadvantageous.
In 2017 approximately 43.6~\% of the world population lived on less than \$5.50 per day, respectively~\cite{PovertyO1:online}.
People that cannot afford air conditioning due to limited financial means, and have limited access to electricity and water, should certainly not be discouraged from using fans when \ac{t-db} exceeds \ac{t-sk}.
Similarly, even in upper-middle, and high-income countries, where access to electricity is less of a concern, low-income people, the elderly, and people with varying types of physical disabilities are those who are most at risk during heatwaves and may not be able to cool or leave their homes during heatwaves.
Hence, they too should not be discouraged from using electric fans.
Elevating air speed indoors would have the additional benefit of reducing the peak energy demand during hot days and the burden on the electric grid while reducing global greenhouse gas emission.

Talk also about:
\begin{itemize}
    \item Talk about dehydration and the fact that with electric fans \ac{h-e} increases, hence the \ac{m-sweat} does not significantly increase. % todo plot that
\end{itemize}

\section{Conclusions}\label{sec:conclusions}
% todo write conclusions
\begin{itemize}
    \item fans can be used at high temperatures and high enthalpies, so several current health guidelines should be revised to allow that.
    \item we have assembled and validated an improved model of heat stress physiology that is sensitive to air-movement cooling and put it in the open domain.
\end{itemize}


\section*{Limitations}
Our heat balance model uses coefficients that were estimated empirically and some simplified equations (e.g., to calculate the respiratory losses) were used.
Consequently, results may not be applicable to all individuals, such as those who are under medications or have pre-existing conditions.
