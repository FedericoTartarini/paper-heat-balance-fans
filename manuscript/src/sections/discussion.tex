%! Author = sbbfti
%! Date = 22/10/2020

\section{Discussion}\label{sec:discussion}

Electrical fans are a cheaper and more efficient alternative to compressor-based air condition.
They consume less energy than the latter, have lower operational and maintenance cost, and do not use refrigerants who can harm the environment.
However, most public health guidelines discourage the use of elevated air speeds during heatwaves and extreme temperature events.
For example, the \ac{who} discourages people from using fans when dry-bulb temperatures exceed 35~°C~\cite{WMO2015}.
Our results show that this recommendation is inaccurate since despite the fact that sensible heat gains increase as air temperatures exceed the skin temperature, elevated air speed facilitate the amount of latent heat that the body can dissipate towards the environment.
Increasing the average air speed in the space from 0.2 m/s to 0.8 m/s leads to an average increase of 2.0 (0.8, 1.3, 2.5)~°C [mean, (SD, Q1, Q3)] in the temperature above which heat strain would occur, irrespectively of relative humidity.
The use of electrical fans would be detrimental only for very high values of relative humidity and temperature.
For example, electrical fans should not be used when indoor conditions exceed \ac{t-op}~=~51~°C and \ac{rh}~=~50~\% or \ac{t-op}~=~41.6~°C and \ac{rh}~=~80~\%.
These conditions are so extreme that should be avoided in the first place and not even healthy and young individuals should be exposed to them for a prolonged period of time.
Our results validate and extend those previously obtained by \mycite{Jay2015}, who developed a simplified heat model to show that elevated air movements are beneficial even if air temperature exceeds skin temperature.
These findings are extremely relevant considering that it is estimated that in 2017 approximately 9.2~\%, 24.1~\% and 43.6~\% of the world population lived with less than \$1.90, \$3.20 and \$5.50 per day, respectively.
This number is expected to increase in 2020 due to the COVID-19 pandemic,a dn global warming is also estimated to play a negative role~\cite{PovertyO1:online}.
Most of these people live in Sub-Saharan Africa and South Asia, regions which are characterized by high temperatures and where climate change may be an acute threat~\cite{PovertyO1:online}.
While, arguably these people may not even have the financial means to even buy electrical fans or may not have access to electricity, however, they should not be discouraged to use fans when temperatures exceed 35~°C.
Also because they may be living in rural area and do not have the opportunity of find shelter from heatwaves in publicly air-conditioned places.
Similarly in developed nations, where most of the people have access to electricity, poor people, the elderly and people with varying types of physical disabilities are those who are mostly at risk during heatwaves, and may not be able to cool their homes during heatwaves.
While electrical fans during heat waves may not alone be able to protect people from physiological strain, they still provide some benefits, as supported by existing literature~\cite{Jay2015, Jay2019a}.

\begin{itemize}
    \item Say there are no models who allow to estimate the benefit of elevated air speed under various environmental and personal factors.
    \item Critically discuss guidelines in terms of max temperatures
    \item Talk about dehydration and the fact that with electrical fans \ac{h-e} increases, hence the \ac{m-sweat} does not significantly increases. % todo plot that
    \item List other strategies that could be used by people like evaporative cooling, ingestion of cold drinks and skin wetting.
\end{itemize}

Limitations:
\begin{itemize}
    \item We used an heat balance models which coefficients that were estimated empirically, hence, they may not be applicable to all individuals.
    Moreover, some of the heat losses are determined using simplified equations (e.g., the respiratory losses).

    \item Critically discuss guidelines in terms of max temperatures
    \item Talk about dehydration and the fact that with electrical fans \ac{h-e} increases, hence the \ac{m-sweat} does not significantly increases. % todo plot that
    \item List other strategies that could be used by people like evaporative cooling, ingestion of cold drinks and skin wetting.
\end{itemize}
