%! Author = sbbfti
%! Date = 22/10/2020

\section{Discussion}\label{sec:discussion}

Electrical fans are a cheaper and more efficient alternative to compressor-based air condition.
They consume less energy than the latter, have lower operational and maintenance cost, and do not use refrigerants who can harm the environment.
However, most public health guidelines discourage their use during heatwaves and extreme temperature events.
For example, the \ac{who} discourages people from using fans when \acf{t-db} exceed 35~$^{\circ}$C~\cite{WMO2015}.
Our results show that this recommendation is inaccurate.
While sensible heat gains increase as the dry-bulb air temperature exceeds the skin temperature, elevated air speed facilitate the amount of latent heat that the body can dissipate towards the environment.
Increasing the average air speed in the space from 0.2 m/s to 0.8 m/s leads to an average increase of 2.0 (0.8, 1.3, 2.5)~$^{\circ}$C [mean, (SD, Q1, Q3)] in the temperature above which heat strain would occur, irrespectively of relative humidity.
The use of electrical fans would be detrimental only for very high values of relative humidity and temperature.
For example, electrical fans (\ac{v}~=~0.8~m/s) should not be used when indoor conditions exceed \ac{t-op}~=~51~$^{\circ}$C and \ac{rh}~=~60~\% or \ac{t-op}~=~44~$^{\circ}$C and \ac{rh}~=~80~\%.
These conditions can be so demanding on the human body that not even healthy and young individuals should be exposed to them.
In addition, based on the climatic data obtained, it appears that these conditions are extremely rear and have not occurred over the last 10 years, and are more extreme than the conditions recorded during the deadliest heatwaves recorded in recent history with the~\cite{WMO2015}.
% todo cite the who database instead
Our results validate and extend the findings of \mycite{Jay2015}, who developed a simplified heat model to show that elevated air movements are beneficial even if \ac{t-db} exceeds \acf{t-sk}.
In addition, these findings are relevant considering that it is estimated that in 2017 approximately 9.2~\%, 24.1~\% and 43.6~\% of the world population lived with less than \$1.90, \$3.20 and \$5.50 per day, respectively.
This number is expected to increase in 2020 due to the COVID-19 pandemic, and global warming is also estimated to increase frequency and intensity of heatwaves~\cite{PovertyO1:online}.
Most of these people live in Sub-Saharan Africa and South Asia, regions which are characterized by high temperatures and where climate change may be an acute threat~\cite{PovertyO1:online}.
While, arguably these people may not even have the financial means to buy electrical fans or do not have access to electricity, they should certainly not be discouraged to use fans when temperatures exceed 35~$^{\circ}$C\@.
In addition, poorer people may be living in rural area and do not have the opportunity to shelter themselves from heatwaves in publicly air-conditioned places.
Similarly, in developed nations, where access to electricity is less of a concern, poor people, the elderly and people with varying types of physical disabilities are those who are mostly at risk during heatwaves, and may not be able to cool their homes during heatwaves.
Hence, they too should not be discouraged from using electrical fans.
Electrical fans during severe heat waves may not alone be able to protect people from physiological strain, they still provide some benefits, as supported by existing literature~\cite{Jay2015, Jay2019a}.
Electrical fans can consequently be used in combination with conventional compressor-based air conditioning to increase comfort conditions indoors while increasing temperature set-points.
% todo say something about heat wave in russia, it was not that hot but people died anyway

\begin{itemize}
    \item Say there are no models who allow to estimate the benefit of elevated air speed under various environmental and personal factors.
    \item Critically discuss guidelines in terms of max temperatures
    \item Talk about dehydration and the fact that with electrical fans \ac{h-e} increases, hence the \ac{m-sweat} does not significantly increases. % todo plot that
    \item List other strategies that could be used by people like evaporative cooling, ingestion of cold drinks and skin wetting.
\end{itemize}

Limitations:
\begin{itemize}
    \item We used an heat balance models which coefficients that were estimated empirically, hence, they may not be applicable to all individuals.
    Moreover, some of the heat losses are determined using simplified equations (e.g., the respiratory losses).
\end{itemize}
