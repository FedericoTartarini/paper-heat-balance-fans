%! Author = sbbfti
%! Date = 22/10/2020

\section{Discussion}\label{sec:discussion}

Electrical fans are a cheaper and more efficient alternative to compressor-based air condition.
They consume less energy than the latter, have lower operational and maintenance cost, and do not use refrigerants who can harm the environment.
However, several public health guidelines discourage their when \acf{t-db} exceed 35~$^{\circ}$C~\cite{WMO2015}.
Our results show that this recommendation is inaccurate and misleading.
While sensible heat gain increases as the \ac{t-db} exceeds \acf{t-sk}, elevated air speeds increase by a greater amount the latent heat that the body can dissipate towards the environment.
This is true for more than 99~\% extreme heat events recorded worldwide.
Increasing the average air speed in the space from 0.2 m/s to 0.8 m/s leads to an average increase of 1.7 (0.8, 1.4, 2.1)~$^{\circ}$C [mean, (SD, Q1, Q3)] in the temperature above which heat strain would occur.
In very hot and dry conditions (i.e., \ac{t-db}~=~47~$^{\circ}$C, \ac{rh}~=~15~\%) fans should only be used in combination with evaporative cooling technologies to prevent heat strain from occurring.
On the other hand we found that people should not use electrical fans (e.g., \ac{v} higher than 0.8~m/s) when indoor conditions exceed \ac{t-op}~=~50~$^{\circ}$C and \ac{rh}~=~40~\%, \ac{t-op}~=~40~$^{\circ}$C and \ac{rh}~=~82~\% or \ac{t-op}~=~37.5~$^{\circ}$C and \ac{rh}~=~100~\%.
It should be noted that such extreme weather events are rare and may only occur in few locations across the world~\cite{ASHRA2017}.
We also compared these data with the records available in the Emergency Events Database (EM-DAT)~\cite{EMDATThe70:online}.
A total of 122 heatwave events in the EM-DAT had information on the maximum air temperature recorded.
These heatwaves were the cause of approximately 117000 deaths, out of which a total of 102876 and 3803 people died during heatwaves with maximum temperatures lower than 45~$^{\circ}$C and 40~$^{\circ}$C, respectively.
During heatwaves as \ac{t-db} increases \ac{rh} generally decreases, hence, it may be hypothesized than the use of electrical fans would have been beneficial in most of those scenarios.

Our results validate and extend the findings of \mycite{Jay2015}, who developed a simplified heat model to show that elevated air movements are beneficial even if \ac{t-db} exceeds \acf{t-sk}.
While the use of compressor-based cooling would arguably be more effective than electric fans in ensuring that people do not suffer from heat strain during heatwaves, it should be noted that in 2017 approximately 9.2~\%, 24.1~\% and 43.6~\% of the world population lived with less than \$1.90, \$3.20 and \$5.50 per day, respectively.
This number is expected to increase in 2020 due to the COVID-19 pandemic, and global warming is also estimated to increase frequency and intensity of heatwaves~\cite{PovertyO1:online}.
Sub-Saharan Africa and South Asia are the poorest regions of the world.
They are characterized by high temperatures and is where climate change is expected to be an acute threat for the whole population~\cite{PovertyO1:online}.
These people who may have limited financial means, access to electricity, and water, should certainly not be discouraged to use fans when temperatures exceed 35~$^{\circ}$C\@.
In addition, poor people living in rural area may not have the opportunity to shelter themselves from heatwaves in publicly air-conditioned places.
Similarly, in developed nations, where access to electricity is less of a concern, low income people, the elderly and people with varying types of physical disabilities are those who are mostly at risk during heatwaves, and may not be able to cool or leave their homes during heatwaves.
Hence, they too should not be discouraged from using electrical fans.

During severe heatwaves, electrical fans may not alone be able to protect people from physiological strain, however, in most locations they would still provide some benefits, albeit marginal.
All individuals should ensure that they keep themselves hydrated by drinking plenty of water, take cool showers or baths, wear light loose clothing, avoid physical activity, eat small cold meals, move to cooler areas of the home and do not expose themselves to direct solar radiation~\cite{HeatandH28:online}.
Electrical fans can be used in combination with conventional compressor-based air conditioning to increase comfort conditions indoors while increasing temperature set-points.

\begin{itemize}
    \item Say there are no models who allow to estimate the benefit of elevated air speed under various environmental and personal factors.
    \item Talk about dehydration and the fact that with electrical fans \ac{h-e} increases, hence the \ac{m-sweat} does not significantly increases. % todo plot that
    \item List other strategies that could be used by people like evaporative cooling, ingestion of cold drinks and skin wetting.
\end{itemize}

In this manuscript we used an heat balance model that uses coefficients that were estimated empirically and some simplified equations (e.g., to calculate the respiratory losses) were used.
Consequently, results may not be applicable to all individuals, such as those who are under medications or have pre-existing conditions.
