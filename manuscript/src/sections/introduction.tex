%! Author = sbbfti
%! Date = 10/06/2020

\section{Introduction}\label{sec:introduction}

Anthropogenic activities are the primary causes of global warming.
From the pre-industrial period human activities are estimated to be the primary cause of a 1~$^{\circ}$C increase in the Earth's global average temperature~\cite{GlobalWa91:online}.
Since 1981 the combined land and ocean temperature has increased at an average rate of 0.18~$^{\circ}$C per decade~\cite{GlobalCl28:online}.
As a consequence, nine of the ten warmest years on records have occurred since 2005~\cite{ClimateC26:online}.
It is estimated that by 2030 the rate at which the global average temperature increases is going to accelerate due to the heating imbalance between the greenhouse gases and the oceans' thermal inertia~\cite{ClimateC26:online}.
Consequently, due to global warming, associated risk to rising temperatures such as heatwaves are expected to increase in length, intensity and frequency~\cite{Whatharm75:online}.

Because of the wide range of impact that heatwaves have across different sectors, there is not an universal definition of heat waves~\cite{Perkins2013}. \ac{wmo} and \ac{who} describe heatwaves as: ``a periods of unusually hot and dry or hot and humid weather that have a subtle onset and cessation, a duration of at least two or three days, usually with a discernible impact on human and natural systems"~\cite{WMO2015}.
Some thermoregulatory factors, demographics and socioeconomic characteristics, such as age (very young and elderly), pre-existing conditions, low income, homeless, prolonged outdoor activities, and social isolation all play a negative role and increase the heat risk during heat waves~\cite{WMO2015}.
Heatwaves also put an additional burden on the health, emergency, energy, water and transportation sectors.
For example, during heat waves peak demand consumption increases due to an increase energy demand for cooling, while the efficiency of the grid and the power plant decreases.
The \ac{iea} estimates that in 2016 in some regions, such as Middle East and some parts of the US, during heatwaves space cooling accounts for more than 70 \% of the residential energy consumption~\cite{IEA2018}.
This may result in power shortages and blackouts.

Both humid and dry heat conditions can affect the heat balance of the human body.
Rapid changes in heat gains can compromise the ability of the body to regulate its core temperature and can worsen health of those with pre-existing conditions or result in illnesses which may be life threatening~\cite{WMO2015}.
The \ac{who} and the \ac{wmo} together with national government agencies provide public health guidance for people to minimise heat stress during heatwaves.
For example, they suggest to keep the body hydrated, to keep out of the heat and keep the home cool.
However, the \ac{who} also state that if the \ac{t-db} is higher than 35~$^{\circ}$C fans can make an individual hotter and ``fans should be discouraged unless they are bringing in significantly cooler air''~\cite{WMO2015}.
They also mention that when \ac{t-db} is higher than 35~$^{\circ}$C, fans may not prevent heat-related illness~\cite{HeatandH28:online}.
Ready.gov, a national public service campaign of the U.S. Government aimed to "educate and empower the American people to prepare for, respond to and mitigate emergencies, including natural and man-made disasters", states that electrical fans should not be used when outside temperatures are higher than 35~$^{\circ}$C.
Finally the \ac{cdc} states that when temperature are in the high 90s (i.e., above 32~$^{\circ}$C) fans will not prevent heat-related illness~\cite{ExtremeH66:online}.
The \ac{epa} Excessive Heat Events Guidebook discourages to direct the flow of fans towards the body when \ac{t-db} is higher than 32.2~$^{\circ}$C~\cite{UnitedStatesEnvironmentalProtectionAgency2006}.
The above recommendations seem to underestimate the benefit of using electrical fans despite the fact that previous research has shown that healthy adults would benefit from their use even if \ac{t-db} is higher than the \ac{t-sk}~\cite{Rate2015, Jay2015, Jay2019a, Rate2015, Gagnon2017}.
The use of electrical fans is still beneficial, even when \ac{t-db} exceeds \ac{t-sk}, since they increase the amount of heat that the body can dissipate towards its environment via the evaporation of sweating~\cite{Jay2015}.
It should be noted, that the evaporative heat loss varies as a function of the \ac{rh} in the space.
Hence, the maximum \ac{t-db} at which electrical fans should be used varies as a function of \ac{rh}, as shown in several experimental studies listed below.

\mycite{Rate2015} conducted a laboratory experiment comprising eight healthy males and concluded that electric fans help in preventing heat-related elevation in heart rate and \ac{t-cr} in both of the following conditions \ac{t-db}~=~42~$^{\circ}$C and \ac{rh}~=~50~\%, and \ac{t-db}~=~36~$^{\circ}$C and \ac{rh}~=~80~\%.
\mycite{Morris2019} also obtained similar results in an experimental study conducted in laboratory settings.
Their experiment, comprising 12 healthy men, concluded that when \ac{t-db}~=~47~$^{\circ}$C and \ac{rh}~=~15~\% fan use is not advisable while when \ac{t-db}~=~40~$^{\circ}$C) and \ac{rh}~=~51~\%) is, in this condition fans reduced core temperature and cardiovascular strain and improved thermal comfort~\cite{Morris2019}.
It should be noted that in the latter condition the enthalpy of the air was higher (101.0~kJ/kg) than in the former condition (73.0~kJ/kg).
In the case with \ac{rh}~=~15~\% the heat strain arguably occurs since all the sweat can easily evaporate even without the use of elevated air movement and the increased air movement increases the convective heat gains.
In hot and dry conditions skin wetting or evaporative cooling technologies can be used to either increase latent heat loss from the skin or to reduce \ac{t-db} and consequently reducing the sensible heat gains, respectively. 
Evaporative cooling is an isoenthalpic process that causes a drop in \ac{t-db} proportional to the sensible heat drop and an increase in humidity ratio proportional to the latent heat gain.
For example, evaporative cooling could have been used to economically and efficiently cool the air from \ac{t-db}~=~47~$^{\circ}$C and \ac{rh}~=~15~\%) to \ac{t-db}~=~40~$^{\circ}$C and \ac{rh}~=~28~\%\@. 
At such conditions, the heat stress on occupants would be even lower than the condition tested in \mycite{Morris2019} experiment and consequently participants would not have experienced heat strain.
Electrical fan use also affects young and aged adults differently.
For example, \mycite{Gagnon2017} exposed nine young ($26 \pm 3$ yr; range 21--30, five men and four women) and nine aged (68 $\pm$ 3 yr; range 61--72, three men and six women) healthy adults to \ac{t-db}~=~42~$^{\circ}$C and \ac{rh} ranging from 30~\% to 70\%.
They observed that fan use resulted in a greater heart rate and \ac{t-cr} in the aged, but not in the young.
It should be noted that the temperature tested in this study was significantly higher than \ac{t-sk}.
More experimental research is, therefore, needed to better estimate how electrical fan use would affect aged adults when exposed to temperatures ranging from 35~$^{\circ}$C and 42~$^{\circ}$C. 

Based on the available scientific evidence, advising healthy adults not to use fans when \ac{t-db} exceed \ac{t-sk} or 35~$^{\circ}$C could, therefore, be negative for many of them who may neither have the resources to cool their homes using air conditioning, nor the opportunity to go to air conditioned public spaces.
The people that may not have access to air conditioning may also be the one that have other burdens that could increase risk, and discouraging them from utilizing electrical fans when temperatures exceed 35~$^{\circ}$C can exacerbate heat stress.

The \ac{iea} estimated that in 2016 approximately 2.8 billion people lived in the hottest parts of the world, and only 8 \% of them had compressor-based air conditioners installed in their homes, compared to 90 \% ownership in the US and Japan~\cite{IEA2018}.
However, the former percentage is estimated to grow rapidly in the coming years as income levels rise.
While this will increase the standards of living, it will also put significant pressure on electricity grids and global greenhouse gas emission~\cite{IEA2018}.
Electrical fans are relative inexpensive to buy (for example,~10 USD in India), energy efficient, some (e.g., pedestal and desk) do not have any installation cost and can consume as low as few Watts of power to operate if they use DC motors~\cite{Yang2015a}.
Electrical fans can, therefore, be used as an alternative cooling technology or in combination with compressor-based air conditioning~\cite{Jay2019a, Hoyt2015, Schiavon2008}.
Previous research has also shown that they can even be used in tropical hot and humid climates to increase occupants satisfaction with their thermal environment~\cite{Lipczynska2018a}.
Elevating air speed indoors has also the potential of reducing the peak energy demand during hot days, the burden on the electrical grid and reducing global greenhouse gas emission.

In 2016, an estimated 2.3 billion electrical fans were in use worldwide and they remain the most common form of cooling in household, consuming approximately only 10 \% of the energy used by a compressor based air conditioner~\cite{IEA2018}.
It is estimated that one in two household owns at least one fan and there are twice as many fans as air conditioners in use worldwide, however, this ratio is declining rapidly~\cite{IEA2018}.
In 2016, the \ac{iea} estimated that electrical fans accounted for 1.5 \% of the total residential energy consumption worldwide (i.e., 80 TWh of energy), and their energy use has increased 3.6 times between 2000 and 2016~\cite{IEA2018}.
The overall energy consumption could, however, sharply increase if people start replacing them with air conditioners.
Moreover, air conditioners not only significantly consume more energy than electrical fans but also can negatively impact climate change when refrigerants are leaked in the atmosphere~\cite{IEA2018}.

To better understand under which environmental conditions (i.e., \ac{t-db}, \ac{rh}) the use of fans would be beneficial, \mycite{Jay2015} developed a simplified heat balance model.
However, their model could be improved because it does not: allow user to change the value of \ac{met} or \ac{clo}, estimates iteratively \ac{t-sk} and \ac{t-cr}, consider radial asymmetry, and their results are based on a single air speed velocity 4.5~m/s which cannot be easily achieved by most of ceiling fans~\cite{Raftery2019}, and pedestal fans~\cite{Yang2015a}.
The value of 4.5~m/s was measured at one meter away from the fan, hence, their results are difficult to generalise.
Consequently, to overcome the above mentioned limitations, we are proposing the use of the heat balance model developed by~\mycite{GaggeSET} to estimate heat losses and physiological variables as a function of environmental and personal factors~\cite{Gagge1986}.
\mycite{GaggeSET} model was originally developed to calculate the \ac{set}.
The \mycite{GaggeSET} can, however, be used to estimate the combinations of \ac{t-db}, \ac{rh}, \ac{t-r}, \ac{v}, \ac{clo}, \ac{met} at which the use of elevated air speed would be beneficial during heatwaves.
In addition, to help the overall community we developed an open-source, free to use, web-based online tool which provides interactive plots and displays the results to the user.
Our tool allows users to determine when people can safely use elevated air speeds to cool themselves.