%! Author = sbbfti
%! Date = 10/06/2020

% Document

\section*{Questions}

\begin{itemize}
    \item Shall we consider the effect of humidifying the outdoor air to cool it and reduce the thermal strain due to increased sweating?
    This could be a solution to the problem that sweat rate is limited to 500mL/h
    \item Which journal shall we target?
    \item Shall we provide results for an older adult as well?
    If so how will we calculate the w for the elderly?
    \item Are we planning to conduct any study in the lab with people, do we have a thermal manikin to estimate latent heat losses?
    \item Shall we look at the weather model and estimate how the prevalence of hot days will change in the next 50 years or so?
\end{itemize}

\section{Introduction}\label{sec:introduction}

Discuss and cover the following points.

\begin{itemize}
    \item Global warming is affecting the climate and most of the hottest years on record have been recorded in the last few decades.
    \item A great number of people live in the tropical and dry climates.
    This regions inhabited by low income people who may not be able to afford to pay for air-conditioning.
    \item Electric fans are an economical and efficient solution to the problem.
    \item Fans increase the air speed around the human body.
    If the air temperature is higher than the skin temperature, then the air movement causes an heat gain.
    However, air movement also increases significantly the potential for evaporation, which in turn is affected by the relative humidity in the space.
    If the latter is higher than then sensible heat gains, then the use of fans is positive.
    However, it should be considered that the potential greater requirement for sweat production with the use of fans can cause dehydration and cardiovascular strain.
    \item Review of international standards on the use of air movement during hot days.
    \item Review of literature Gupta 2012, and describe issues with Ollie 2015 model.
    \begin{enumerate}
        \item only focuses on air temperature and relative humidity. Clothing (summer clothing), metabolic rate (1.1 met), and mean radiant temperature (equal to air tmp) are constant.
        \item is model only works for one type of fan 18", at max speed (4.5 m/s) and placed at 1 m from the occupant.
        \item skin temperature is considered constant and equal to 35.5 C
        \item body surface area is also considered constant = 1.8 m2
        \item he uses met of a standing person but radiative area of sitting one.
        \item because of the previous assumptions even the evaporative heat transfer coefficient is considered to be constant.
    \end{enumerate}
    \item The aim of this manuscript is to determine at which combinations of environmental parameters (i.e., air temperature, relative humidity, mean radiant temperature and air velocity) and personal parameters (i.e., clothing, metabolic rate and body surface area) the use of fan is beneficial.
    We, therefore, used the standard effective temperature model (SET) heat balance equations to determine the set of environmental and personal conditions above which the use of fans would be detrimental.
    We also estimated the environmental limits above which elevated cardiovascular strain and thermal strain would occur.
    We determined the predicted sweat losses and estimated the risk of dehydration.
    Finally, we developed an open source, free to use, web-based online tool which can be used to estimate all the aforementioned results as a function of environmental and personal variables.
\end{itemize}





