%! Author = sbbfti
%! Date = 10/06/2020

% Document

%\section*{Questions} todo
%
%\begin{itemize}
%    \item Shall we consider the effect of humidifying the outdoor air to cool it and reduce the thermal strain due to increased sweating?
%    This could be a solution to the problem that sweat rate is limited to 500mL/h
%    \item Which journal shall we target?
%    \item Shall we provide results for an older adult as well?
%    If so how will we calculate the w for the elderly?
%    \item Are we planning to conduct any study in the lab with people, do we have a thermal manikin to estimate latent heat losses?
%    \item Shall we look at the weather model and estimate how the prevalence of hot days will change in the next 50 years or so?
%\end{itemize}

\section{Introduction}\label{sec:introduction}

Anthropogenic activities are the primary causes of global warming.
From the pre-industrial period human activities are estimated to be the primary cause of a 1~$^{\circ}$C increase in the Earth's global average temperature~\cite{GlobalWa91:online}.
Since 1981 the combined land and ocean temperature has increased at an average rate of 0.18~$^{\circ}$C per decade~\cite{GlobalCl28:online}.
As a consequence, nine of the ten warmest years on records have occurred since 2005~\cite{ClimateC26:online}.
It is estimated that by 2030 the rate at which the global average temperature increases is going to accelerate due to the heating imbalance between the greenhouse gases and the oceans' thermal inertia~\cite{ClimateC26:online}.
Consequently, due to global warming, associated risk to rising temperatures such as heatwaves are expected to increase in length, intensity and frequency~\cite{Whatharm75:online}.

While there is not an universally definition of heatwave both the \ac{wmo} and \ac{who} describe them as: ``a periods of unusually hot and dry or hot and humid weather that have a subtle onset and cessation, a duration of at least two or three days, usually with a discernible impact on human and natural systems"~\cite{WMO2015}.
Some thermoregulatory factors, demographics and socioeconomic characteristics, such as age (very young and elderly), pre-existing conditions, low income, prolonged outdoor activities, and social isolation all play a negative role and increase the heat risk of an individual during heat waves~\cite{WMO2015}.
Heatwaves also put an additional burden on the health, emergency, energy, water and transportation sectors.
For example, during heat waves peak demand consumption increases due to an increase energy demand for cooling, while the efficiency of the grid and the power plant decreases.
This may result in power shortages and blackouts.

Both humid and dry heat conditions can affect the heat balance of the human body.
Rapid changes in heat gains can compromise the ability of the body to regulate its core temperature and can worsen health of those with pre-existing conditions or result in illnesses which may even be life threatening~\cite{WMO2015}.
The \ac{who}, and the \ac{wmo} together with national government agencies provide public health guidance for people to minimise heat stress during heatwaves.
For example, they suggest to keep the body hydrated, to keep out of the heat and keep the home cool.
However, the \ac{who}, and several U.S.A.\ government agencies do not recommend the use of electric fan when temperatures exceed 35~$^{\circ}$C~\cite{ExtremeH66:online, Frequent18:online, HeatandH28:online, WMO2015}, despite the fact that previous research has shown that the use of electrical fans is beneficial even if the \ac{t-db} is higher than the \ac{t-sk}~\cite{Jay2015, Jay2019a, Rate2015}.
For example, the \ac{epa} Excessive Heat Events Guidebook discourages the use of electrical fans during heat waves to cool people, but instead it recommends their use to bring cooler air from outside~\cite{UnitedStatesEnvironmentalProtectionAgency2006}.
% guidelines in canada https://ncceh.ca/content/fans
While the latter in principle it is a good advice, natural ventilation is beneficial only when the enthalpy of air outdoors than indoors air~\cite{Fiorentini2019}\@.
During heatwaves people may not be fully aware of the weather conditions outdoors and heatwaves may keep their windows open during the hottest hours of the day which in turn would lead to heat gains into the building~\cite{Tartarini2017}.
One laboratory study conducted on 12 healthy men concluded that in hot (\ac{t-db}~=~47~$^{\circ}$C) and dry conditions (\ac{rh}~=~15~\%) fan use is not advisable while in hot (\ac{t-db}~=~40~$^{\circ}$C) and humid conditions (\ac{rh}~=~51~\%) fans reduced core temperature and cardiovascular strain and improved thermal comfort~\cite{Morris2019}.
It should, however, be noted that in the latter condition the enthalpy of the air was higher (101.0~kJ/kg) than in the former condition (73.0~kJ/kg).
In the hot and dry condition heat strain arguably occurs since all the sweat can easily evaporate even without the use of elevated air movement.
Consequently, in hot and dry conditions skin wetting or evaporative cooling technologies can be used to either increase latent heat losses from the skin or to slightly reduce \ac{t-db} and consequently reducing the sensible heat gains, respectively.
In the latter case the use of elevated air speed can then be used to increase the evaporate heat transfer coefficient.
Either of the above mentioned strategies would reduce the heat strain on the human body.
For example, evaporative cooling could have been used to cool the air in the hot and dry scenario presented above (\ac{t-db}~=~47~$^{\circ}$C, \ac{rh}~=~15~\%) to the following condition \ac{t-db}~=~40~$^{\circ}$C, \ac{rh}~=~28~\%\@.
In the latter condition the heat strain on occupants would be even lower than the condition tested in \mycite{Morris2019} experiment and consequently participants would not have experienced heat strain.

Based on the available scientific evidence, advising people not to use fans when temperatures exceed 35~$^{\circ}$C could, therefore, be detrimental for many people around the world since many of them may neither have the resources to cool their space using air conditioning, nor the ability to travel to public spaces which are cooler.
Some of these people could be the elderly, people with mobility issues or poor people living either in remote areas or in developing parts of the world.
These groups are already more at risk than the rest of the population, and discouraging them from utilizing electrical fans when temperatures exceed 35~$^{\circ}$C can exacerbate heat stress.
\mycite{Jay2015} demonstrated that the use of fans is beneficial even if \ac{t-db} is higher than \ac{t-sk}, as long as the increase in \ac{e-sk} can compensate for the extra sensible heat gains~\cite{Jay2015}.

Electrical fans are relative inexpensive to buy (can be purchased for 20 USD), energy efficient, do not have any installation cost and can consume as low as few Watts of power to operate.
Electrical fans can, therefore, be used as an alternative cooling technology or in combination with compressor-based air conditioning~\cite{Jay2019a, Yang2015a}.
Previous research has also shown that they can even be used in tropical hot and humid climates to increase occupants satisfaction with their thermal environment~\cite{Lipczynska2018a}.
Therefore, elevating air speed indoors has also the potential of reducing the peak energy demand during hot days and the burden on the electrical grid.

To better understand under which environmental conditions (i.e., \ac{t-db}, \ac{rh}) the use of fans would be beneficial, \mycite{Jay2015} developed a simplified heat balance model.
However, their model had the following limitations, it does not: consider radial asymmetry, estimates iteratively \ac{t-sk} and \ac{t-cr}, allow user to change the value of \ac{met} or \ac{clo}.
Moreover, their results are based on a single air speed velocity 4.5~m/s which cannot be achieved by most of ceiling fans~\cite{Raftery2019}, and pedestal fans~\cite{Yang2015a}.
The value of 4.5~m/s was measured at one meter away from the fan, hence, their results are difficult to generalise.
Consequently, to overcome the above mentioned limitations, we are proposing the use of the heat balance model developed by~\mycite{GaggeSET} to estimate heat loss and physiological variables as a function of environmental and personal factors.
\mycite{GaggeSET} model was originally developed to calculate the \ac{set}.
The \mycite{GaggeSET} can, however, be used to estimate the combinations of \ac{t-db}, \ac{rh}, \ac{t-r}, \ac{v}, \ac{clo}, \ac{met} at which the use of elevated air speed would be beneficial during heatwaves.
In addition, to help the overall community we developed an open-source, free to use, web-based online tool which provides interactive plots and displays the results to the user.
Our tool allows users to determine when people can safely use elevated air speeds to cool their body.

%\begin{itemize}
%    \item I do not really see the benefit of the model proposed by Hosper since: 1) they set the limit of sweating to 440 mL/h; and they limit 2) the amount of water than I person can spray on him self to 116 mL/h. This would not affect the results in the SET model since with the SET model the estimated sweating rate is lower.
%    Very dry climates could benefit from using evaporative cooling.
%\end{itemize}