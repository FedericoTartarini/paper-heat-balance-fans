%! Author = sbbfti
%! Date = 10/06/2020

\section{Introduction}\label{sec:introduction}

Anthropogenic activities are the primary causes of global warming.
From the pre-industrial period, human activities are estimated to be the primary cause of a 1~$^{\circ}$C increase in the Earth's global average temperature~\cite{GlobalWa91:online}.
Since 1981 the combined land and ocean temperature has increased at an average rate of 0.18~$^{\circ}$C per decade~\cite{GlobalCl28:online}.
As a consequence, nine of the ten warmest years on record have occurred since 2005~\cite{ClimateC26:online}.
It is estimated that by 2030 the rate at which the global average temperature increases is going to accelerate due to the heating imbalance between the greenhouse gases and the oceans' thermal inertia~\cite{ClimateC26:online}.
Global warming will cause peak phenomena like heatwaves to increase in length, intensity, and frequency~\cite{Whatharm75:online}.

Heatwaves impact various sectors differently, hence, there is not a universal standardized scientific definition~\cite{Perkins2013}.
The \ac{wmo} and \ac{who} describe heatwaves as: ``periods of unusually hot and dry or hot and humid weather that have a subtle onset and cessation, a duration of at least two or three days, usually with a discernible impact on human and natural systems"~\cite{WMO2015}.
Some thermoregulatory factors, demographics, and socioeconomic characteristics, such as age (very young and elderly), pre-existing health conditions, low income, homeless, prolonged outdoor activities, and social isolation all play a negative role and increase the heat risk during heatwaves~\cite{WMO2015}.
Heatwaves also put an additional burden on the health, emergency, energy, water, and transportation sectors.
For example, during heatwaves peak electricity demand increases due to increased demand for cooling, while the efficiency of the grid and power generation decreases.
The \ac{iea} estimated that in 2016 cooling during heatwaves accounted for more than 70 \% of residential energy consumption in regions such as the Middle East and some parts of the U.S.~\cite{IEA2018}.
Such increased demand can cause power shortages and blackouts.

The \ac{iea} estimated that in 2016 approximately 2.8 billion people lived in the hottest parts of the world, and only 8~\% of them had compressor-based air conditioners installed in their homes, compared to 90~\% ownership in the U.S. and Japan~\cite{IEA2018}.
The former percentage is expected to grow rapidly in the coming years as income levels rise.
While this may increase standards of living, it will also put significant pressure on electricity grids and increase global greenhouse gas emissions~\cite{IEA2018}.
Electric fans are relatively inexpensive (for example, a basic model can be purchased for as low as 10 USD in India), energy-efficient, some (e.g., pedestal and desk) do not have any installation cost, and with direct current motors, they now consume single-digit watts and provide substantial air flows~\cite{Yang2015a}.
Electric fans can therefore be used either as an alternative cooling technology or in combination with a reduced level of compressor-based air conditioning~\cite{Jay2019a, Hoyt2015, Schiavon2008}.
Previous research has also shown that they are effective at increasing occupants' thermal satisfaction even in tropical hot and humid climates~\cite{Lipczynska2018a}.
Elevating indoor air speeds while simultaneously increasing temperature set-points of air conditioning has the potential of reducing the peak energy demand during hot days, reducing the burden on the electric grid, and lowering global greenhouse gas emission.

In 2016, an estimated 2.3 billion electric fans were in use worldwide and they remain the most common form of cooling in households~\cite{IEA2018}.
It is estimated that one in two households own at least one fan and there are twice as many fans as air conditioners in use worldwide, however, this ratio is declining rapidly~\cite{IEA2018}.
In 2016, the \ac{iea} estimated that electric fans accounted for 1.5~\% of total residential energy consumption worldwide (i.e., 80~TWh of energy), and their energy use has increased 3.6 times between 2000 and 2016~\cite{IEA2018}.
The overall energy consumption could, however, sharply increase if people start replacing them with air conditioners.
Air conditioners not only significantly consume more energy than electric fans but also can negatively impact climate change when refrigerants are leaked into the atmosphere~\cite{IEA2018}.

Both humid and dry heat conditions affect the heat balance of the human body.
Rapid changes in heat gains can compromise the ability of the body to regulate its core temperature and can worsen the health of those with pre-existing health conditions or result in illnesses that may be life threatening~\cite{WMO2015}.
The \ac{who} and the \ac{wmo} together with national government agencies provide public health guidance for people to minimize heat stress during heatwaves.
% note - OJ You could probably here provide some information linking heat stress to health, and refer to the fact that heat exposure elicits types of physiological strain other than just thermal. e.g. cardiovascular strain, which can then put people with poor cardiovasular health at risk. I think the main point to emphaise is that it is not just core temperature that matters, as people with often do not die or get ill in heatwaves because of heat stroke.
For example, they suggest keeping the body hydrated and out of the heat, and the home cool.
However, the \ac{who} also states that if the \ac{t-db} is higher than 35~$^{\circ}$C fans can make an individual hotter and ``fans should be discouraged unless they are bringing in significantly cooler air''~\cite{WMO2015}.
They also mention that when \ac{t-db} is higher than 35~$^{\circ}$C, fans may not prevent heat-related illness~\cite{HeatandH28:online}.
Ready.gov, a national public service campaign of the U.S.\@ Government that aims to ``educate and empower the American people to prepare for, respond to and mitigate emergencies, including natural and man-made disasters'', states that electric fans should not be used when outside temperatures are higher than 35~$^{\circ}$C\@.
The \ac{cdc} states that when the temperature is in the high 90's ($^{\circ}$F, i.e., above 32~$^{\circ}$C) fans will not prevent heat-related illness~\cite{ExtremeH66:online}.
Similarly, the \ac{epa} Excessive Heat Events Guidebook discourages directing the flow of fans towards the body when \ac{t-db} is higher than 32.2~$^{\circ}$C~\cite{UnitedStatesEnvironmentalProtectionAgency2006}.
The above recommendations underestimate the cooling effect of using electric fans, ignoring research evidence that healthy adults do in fact benefit from their use when \ac{t-db} is higher than the \ac{t-sk}~\cite{Rate2015, Jay2015, Jay2019a, Rate2015, Gagnon2017}.
% note - OJ Some of these guidelines might actually be misguidedly depriving some people of the only cooling option that they have available. I also think it is important to differentiate between agencies that say that fans do not help versus those who say that fans make things worse (i.e. they should be switched off). I think the latter is most crucial because if we are telling people to turn these devices off when they would be helping, the consequences could be quite considerable
The use of electric fans is beneficial, even when \ac{t-db} exceeds \ac{t-sk}, because fan airflow increases the amount of heat that the body can dissipate to the environment via the evaporation of sweating~\cite{Jay2015}.
This applies even at high \ac{rh}, though the evaporative heat loss is a function of humidity in the space.
% note OJ - Worth  empahsising that this is the potenital for the evaporative heat loss
The maximum \ac{t-db} at which electric fans should be used will depend on the \ac{rh}, as shown in several experimental studies listed below.

\mycite{Rate2015} conducted a laboratory experiment comprising eight healthy males and concluded that electric fans help in preventing heat-related elevation in heart rate and \ac{t-cr} in both of the following conditions \ac{t-db}~=~42~$^{\circ}$C and \ac{rh}~=~50~\%, and \ac{t-db}~=~36~$^{\circ}$C and \ac{rh}~=~80~\%.
\mycite{Morris2019} also obtained similar results in an experimental study conducted in laboratory settings.
Their experiment, comprising 12 healthy men, concluded that when \ac{t-db}~=~47~$^{\circ}$C and \ac{rh}~=~15~\% fan use is not advisable.t
On the other hand, when \ac{t-db}~=~40~$^{\circ}$C and \ac{rh}~=~51~\% fans reduced core temperature and cardiovascular strain and improved thermal comfort~\cite{Morris2019}.
It should be noted that the enthalpy of the air in the latter condition was considerably higher (101.0~kJ/kg) than in the former condition (73.0~kJ/kg).
% note OJ - This is also a novel aspect of this Ypaper relevant to our submission that is currently under review
At 47~$^{\circ}$C and 15~\% \ac{rh} heat strain was a risk because all the sweat could evaporate, even without the use of elevated air movement, and air movement then only increases convective heat gain.
In hot and dry conditions, skin wetting or evaporative cooling of the air can be used to respectively increase latent heat loss from the skin or to reduce \ac{t-db} and its associated sensible heat gains.
% note OJ - We have a paper on skin wetting (without fans) published in JAMA (Morris et al. 2019) which might be a relevant reference here. The focus on potenital reduction in air temp with interventions such as misting is also a novel aspect to this paper
Evaporative cooling is an isoenthalpic process that causes a drop in \ac{t-db} proportional to the sensible heat drop and an increase in humidity ratio proportional to the latent heat gain.
For example, evaporative cooling could have been used to economically and efficiently cool the air from \ac{t-db}~=~47~$^{\circ}$C and \ac{rh}~=~15~\% to \ac{t-db}~=~40~$^{\circ}$C and \ac{rh}~=~28~\%\@.
Under such conditions, the heat stress on occupants would be even lower than the condition tested in \mycite{Morris2019} experiment and consequently, participants would not have experienced heat strain.
\mycite{Gagnon2017} exposed nine young ($26 \pm 3$ yr;\@ range 21--30, five men and four women) and nine aged (68 $\pm$ 3 yr;\@ range 61--72, three men and six women) healthy adults to \ac{t-db}~=~42~$^{\circ}$C, a temperature significantly higher than \ac{t-sk}.
The \ac{rh} was varied from 30~\% to 70\%.
They observed that both heart rate and \ac{t-cr} in the aged, but not in the young, were higher with fan use compared to the baseline condition (i.e., no fan use).
% note - OJ This study was a humidity ramp at 42C that start at 30%RH increasing to 70%RH. It is likely that the conditions were uncompensable right for both old and young from the start so probably not the best protocol to use to identify how fans alter the threshold of compensability
It should be noted that the temperature tested in this study was significantly higher than \ac{t-sk}.
More experimental research is needed to determine how electric fan use affects aged adults in the temperature range from 35~$^{\circ}$C to 42~$^{\circ}$C\@.
% note OJ I think it is fair to say that this needs to be assessed in older adults at other temperatures too, especially in dry environments
Based on this evidence, advising healthy adults not to use fans when \ac{t-db} exceed \ac{t-sk} or 35~$^{\circ}$C could increase their risk of suffering from heat strain.
In addition, guidelines like those provided by Ready.gov that discourage people from using their electric fans based on the outdoor temperature are likely to be highly inaccurate and to remove a viable cooling option from occupants in many situations.
% note OJ - I do not fully understand this comment - is this based on the notion that idoor temperatures are usually cooler than outdoor temps?

To better understand under which environmental conditions (i.e., \ac{t-db}, \ac{rh}) the use of fans would be beneficial, \mycite{Jay2015} developed a simplified heat balance model.
They then used their model to develop a simple chart that can be used to assist public health messaging in explaining to young and older adults when the use of electric fans is beneficial to cool the human body.
Based on their results they concluded that the use of electric fans is beneficial even when \ac{t-db} exceeds \ac{t-sk}.
% note EA Need to describe how the model might have been incorporated in guidelines etc to solve the problems you have outlined so far in the paper. FT I have changed the text to address EA's comment. You can say that the model was too simple to be used for design, and that a new model is being proposed here to rectify that. All the details about M and wind speeds should go into a new background section about modeling, or into the next section, Methods...,
Their model could be improved because it does not: allow the user to change the value of \ac{met} or \ac{clo}, estimate iteratively \ac{t-sk} and \ac{t-cr}, consider radial asymmetry, and their results are based on a single air speed 4.5~m/s which cannot be easily achieved by most of the ceiling fans~\cite{Raftery2019}, and pedestal or desk fans~\cite{Yang2015a}.
% note OJ - This is very relevant for physically active people
% note OJ - This speed was based on measured values with an anemometor of an 18" diameter pedestal fan on the highest speed setting at a distance of 1 m. So somewhat relevant to pedestal fans, but certainly true for ceiling fans and desk fans.
The value of 4.5~m/s was measured at one meter away from the fan, hence, their results are difficult to generalize.
To overcome the above-mentioned limitations, we are proposing the use of the heat balance model developed by~\mycite{GaggeSET} to estimate heat losses and physiological variables as a function of environmental and personal factors~\cite{Gagge1986}.
The model outputs the \ac{set} that allows the combined effects of \ac{t-db}, \ac{rh}, \ac{t-r}, \ac{v}, \ac{clo}, \ac{met} to be reduced to a standard environment.
Among its attributes are modeling the physiology of sweating and the effects of air movement on sensible and evaporative heat transfer.
% note OJ - The back ground calculations are probably quite similar - especially for sweating.
ASHRAE Standard 55 uses this model to determine the effects of elevated air speeds on body heat balance and thermal comfort~\cite{ashrae552017}.
% The model has also been tested in extreme conditions~\cite{ASHRA2017}.
The model has also been tested for its relevance under realistic \ac{v} situations produced by different types of fans, such as partial immersion in the elevated flows, and flows with different turbulence intensity~\cite{Huang2014}.
This makes the SET model appropriate for predicting the thermophysiological effects of elevated air speed would be beneficial during heatwaves.
% note OJ - Determining the real v and the accounting for physiological limits of sweating will be important
In addition, to help policymakers and people worldwide we chose to incorporate the model into an open-source, free to use, web-based online tool which provides interactive plots and displays the results to the user.
Our tool should allow users to determine when people can safely use elevated air speeds to cool themselves.