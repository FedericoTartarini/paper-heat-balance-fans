%! Author = Federico
%! Date = 10/06/2020

\section{Introduction}\label{sec:introduction}

The \ac{wmo} and \ac{who} describe heatwaves as: ``periods of unusually hot and dry or hot and humid weather that have a subtle onset and cessation, a duration of at least two or three days, usually with a discernible impact on human and natural systems"~\cite{WMO2015}.
Heatwaves impact various sectors differently, hence, there is not a universal standardized scientific definition~\cite{Perkins2013}.
Global climate change is predicted to cause peak phenomena like heatwaves to increasing in length, intensity, and frequency~\cite{Whatharm75:online}.
A wide range of factors can increase health risks during heatwaves, including individual physiological differences, demographics (e.g., age), socioeconomic characteristics (e.g., income, homelessness), pre-existing health conditions, prolonged outdoor activities, and social isolation~\cite{WMO2015}.
Heatwaves also place an additional burden on the health, emergency, energy, water, and transportation sectors. 
For example, during heatwaves peak electricity demand increases due to increased demand for cooling, while the efficiency of the grid and power generation decreases.
The \ac{iea} estimated that in 2016 the energy required for cooling during heatwaves accounted for more than 70~\% of residential consumption in regions such as the Middle East and some parts of the U.S.~\cite{IEA2018}.
Such increased demand can cause power shortages and blackouts, which can cripple the water supply system as it happened in Pakistan in 2015.
This heatwave claimed more than 700 lives and hospitals had to cope with an increased number of people suffering from heatstroke and dehydration~\cite{Masood2015}.

The \ac{iea} estimated that in 2016 approximately 2.8 billion people lived in the hottest parts of the world, and only 8~\% of them had compressor-based air conditioners installed in their homes, compared to 90~\% ownership in the U.S.\@ and Japan~\cite{IEA2018}.
While the former percentage is expected to grow rapidly in the coming years as income levels rise, it is clear that accessible alternatives to compressor-based air conditioning for personal cooling are an urgent priority~\cite{Davis2015}.
Mass air conditioning use in the future will exert further pressure on electricity grids and exacerbate global greenhouse gas emissions due to the increase in energy consumption~\cite{IEA2018}.
Moreover, the \ac{epa} estimates that globally 25~\% of hydrofluorocarbons with high global warming potential are emitted by residential and commercial air conditioning equipment, and hydrofluorocarbons emissions in developing nations are projected to quadruple by 2030~\cite{Snap2016}.

Increasing air movement is a personalized cooling strategy that bypasses the issues associated with refrigerant gases and is more efficient than compressor-based air conditioning~\cite{Morris2021a}.
Electric fans are relatively inexpensive (for example, a basic model can be purchased for as low as 10 USD in India), energy-efficient, some (e.g., pedestal and desk) do not have any installation cost, and with direct current motors, they now consume single-digit watts and provide substantial air flows~\cite{Yang2015a}.
The relatively low electricity consumption and cost of fans could help mitigating additional negative compounding health factors such as socioeconomic inequality or poor access to air conditioning~\cite{Morris2021a}.
Electric fans can be used either as an alternative cooling technology or in combination with a reduced level of compressor-based air conditioning~\cite{Jay2019a, Hoyt2015, Schiavon2008}.
They are also effective at increasing the thermal satisfaction of occupants even in tropical hot and humid climates~\cite{Lipczynska2018a}.
% Elevating indoor air speeds while simultaneously increasing temperature set-points of air conditioning has the potential of reducing the peak energy demand during hot days, reducing the burden on the electric grid and power outages (which may hinder emergency services), and lowering global greenhouse gas emissions.

In 2016, an estimated 2.3 billion electric fans were in use worldwide and they remain the most common form of cooling in households~\cite{IEA2018}.
It is estimated that one in two households own at least one fan and there are twice as many fans as air conditioners in use worldwide, although this ratio is declining rapidly~\cite{IEA2018}.
In 2016, the \ac{iea} estimated that electric fans accounted for 1.5~\% of total residential energy consumption worldwide (i.e., 80~TWh of energy), and their energy use has increased 3.6 times between 2000 and 2016~\cite{IEA2018}.
The overall household energy consumption could, however, sharply increase if people start replacing them with air conditioners.
Air conditioners consume significantly more energy than electric fans and can negatively impact climate change when refrigerants are leaked into the atmosphere~\cite{IEA2018}.

Excessive heat can compromise the ability to maintain core temperature within safe limits and can worsen the health of those with pre-existing health conditions~\cite{WMO2015}.
For example, people with cardiovascular disease are at an elevated risk of catastrophic collapse due to an exacerbation of cardiovascular strain associated with the maintenance of central blood pressure in the face of rising skin blood flow sometimes accompanied by decreases in blood volume due to dehydration~\cite{Morris2021a}.

The \ac{who} and the \ac{wmo} together with national government agencies provide public health guidance for people to minimize heat stress during heatwaves.
For example, they suggest keeping the body hydrated and out of the heat, and keeping the home cool.
However, the \ac{who} also states that if the \ac{t-db} is higher than 35~$^{\circ}$C fans can make an individual hotter and ``fans should be discouraged unless they are bringing in significantly cooler air''~\cite{WMO2015}.
They also mention that when \ac{t-db} is higher than 35~$^{\circ}$C, fans may not prevent heat-related illness~\cite{HeatandH28:online}.
This is possibly because if \ac{t-db} exceeds \ac{t-sk} (approximately 35~$^{\circ}$C) the gradient for dry heat loss is reversed and sensible heat is added to the body.
Ready.gov, a national public service campaign of the U.S.\@ Government that aims to ``educate and empower the American people to prepare for, respond to and mitigate emergencies, including natural and man-made disasters'', states that electric fans should not be used when outside temperatures are higher than 35~$^{\circ}$C~\cite{ExtremeH66:online}.
According to Ready.gov, in these conditions electric fans could increase the risk of heat-related illness, and that they create airflow and a false sense of comfort, but do not reduce body temperature.
The \ac{cdc} states that when the temperature is in the high 90's ($^{\circ}$F, i.e., above 32~$^{\circ}$C) fans will not prevent heat-related illness~\cite{Frequent18:online}.
Similarly, the \ac{epa} Excessive Heat Events Guidebook discourages directing the flow of fans towards the body when \ac{t-db} is higher than 32.2~$^{\circ}$C~\cite{UnitedStatesEnvironmentalProtectionAgency2006}.

The above recommendations underestimate the evaporative cooling effect of electric fans, ignoring research evidence that healthy adults do in fact benefit from their use when \ac{t-db} is higher than the \ac{t-sk}~\cite{Jay2019a, Rate2015, Jay2015, Gagnon2017} possibly discouraging some people to use the only cooling option that they have available.
It should also be noted that guidelines such as those provided by Ready.gov, may further worsen conditions for many since they do not only state that fans may not help to prevent heat strain but they also encourage people to turn them off when outside temperatures are higher than 35~$^{\circ}$C\@.
This not only encourages people to turn electric fans off when their use is still beneficial but is illogical in that fans should not be controlled as a function of the outdoor temperature but solely on indoor environmental conditions. 
Indoor temperatures are influenced by outdoor conditions but are often different than outdoor temperatures. 
The use of electric fans is beneficial, even when \ac{t-db} exceeds \ac{t-sk}, because fan airflow increases the amount of heat that the body can dissipate to the environment via the evaporation of sweating~\cite{Jay2015}.
This applies even at high \ac{rh} since fans help sweat, which otherwise would either sit on the skin or drip off the body, to evaporate.
The maximum \ac{t-db} at which electric fans should be used will depend on the \ac{rh}, as shown in several experimental studies listed below.

\mycite{Rate2015} conducted a laboratory experiment comprising eight healthy males and concluded that electric fans help in preventing heat-related elevation in heart rate and \ac{t-cr} in both of the following conditions \ac{t-db}~=~42~$^{\circ}$C and \ac{rh}~=~50~\%, and \ac{t-db}~=~36~$^{\circ}$C and \ac{rh}~=~80~\%.
\mycite{Morris2019} also obtained similar results in an experimental study conducted in laboratory settings.
Their experiment, comprising 12 healthy men, concluded that when \ac{t-db}~=~47~$^{\circ}$C and \ac{rh}~=~15~\% fan use is not advisable.
On the other hand, for values of \ac{t-db}~=~40~$^{\circ}$C and \ac{rh}~=~51~\% fans reduced core temperature and cardiovascular strain and improved thermal comfort~\cite{Morris2019}.
It should be noted that the enthalpy of the air in the latter condition was considerably higher (\var{enthalpy_40_51}~kJ/kg) than in the former condition (\var{enthalpy_47_15}~kJ/kg).
At 47~$^{\circ}$C and 15~\% \ac{rh} heat strain was a risk because all the sweat could evaporate, even without the use of elevated air movement, and air movement in those conditions only increases convective heat gain without increasing evaporative heat loss.
In hot and dry conditions, skin wetting or evaporative cooling of the air can be used to respectively increase latent heat loss from the skin or to reduce \ac{t-db} and its associated sensible heat gains.
Skin-wetting has been shown to reduce physiological heat strain, dehydration, and thermal discomfort at temperatures up to 47~$^{\circ}$C, irrespective of \ac{rh}~\cite{Morris2019a}.
Evaporative cooling is an isoenthalpic process that causes a drop in \ac{t-db} proportional to the sensible heat drop and an increase in humidity ratio proportional to the latent heat gain.
For example, evaporative cooling could have been used to economically and efficiently cool the air from \ac{t-db}~=~47~$^{\circ}$C and \ac{rh}~=~15~\% to either \ac{t-db}~=~40~$^{\circ}$C and \ac{rh}~=~\var{rh_t_40_enthalpy_47_15}~\% or to \ac{t-db}~=~30~$^{\circ}$C and \ac{rh}~=~\var{rh_t_30_enthalpy_47_15}~\%\@.
Under such conditions, the heat stress on occupants would be even lower than in the condition tested in \mycite{Morris2019} experiment and consequently, participants would not have experienced heat strain.
\mycite{Gagnon2017} exposed nine young ($26 \pm 3$ yr;\@ range 21--30, five men and four women) and nine aged (68 $\pm$ 3 yr;\@ range 61--72, three men and six women) healthy adults to \ac{t-db}~=~42~$^{\circ}$C, a temperature significantly higher than \ac{t-sk}.
The \ac{rh} was gradually increased throughout the course of the experiment from 30~\% to 70\%.
They observed that both heart rate and \ac{t-cr} in the aged, but not in the young, were higher with fan use compared to the baseline condition (i.e., no fan use) due to the well-described impairments in thermoregulatory sweating with age.
It should be noted that the temperature tested in this study was significantly higher than \ac{t-sk}.
More experimental research is needed to determine how electric fan use affects aged adults in the temperature range from 35~$^{\circ}$C to 42~$^{\circ}$C\@.

Based on this evidence, advising healthy adults not to use fans when \ac{t-db} exceed \ac{t-sk} or 35~$^{\circ}$C could increase their risk of suffering from heat strain and would prevent them from using an effective, energy-efficient, and low-cost cooling technology.
To better understand under which environmental conditions (i.e., \ac{t-db}, \ac{rh}) the use of fans would be beneficial, both \mycite{Jay2015} and \mycite{Morris2021a} developed a simplified heat balance model.
They then used their model to develop a chart that can be used to assist public health messaging in explaining to young and older adults when the use of electric fans is beneficial to cool the human body.
They concluded that the use of electric fans is beneficial even when \ac{t-db} exceeds \ac{t-sk}.
Their model can be further improved allowing the user to change the value of \ac{met} or \ac{clo}, estimate iteratively \ac{t-sk} and \ac{t-cr}, consider radial asymmetry, and adjust air speed.
\mycite{Jay2015} chose a value of 4.5~m/s based on measured values of air speed at a distance of 1.0~m of an 18" pedestal fan set at maximum speed~\cite{Jay2015}.
\mycite{Morris2021a} chose instead a value of 3.5~m/s.
Their results may be difficult to generalize since such high air speeds cannot be easily achieved by most ceiling fans~\cite{Raftery2019}, pedestal, or desk fans~\cite{Yang2015a} available on the market.

To this end, we used the human thermoregulatory model developed by \mycite{GaggeSET} to estimate heat losses and physiological variables as a function of environmental and personal factors~\cite{Gagge1986}.
The model originally outputs the \ac{set} that allows the combined effects of \ac{t-db}, \ac{rh}, \ac{t-r}, \ac{v}, \ac{clo}, \ac{met} to be reduced to a standard environment.
Among its attributes are modeling of the physiology of sweating and the effects of air movement on sensible and evaporative heat transfer.
ASHRAE Standard 55 uses this model to determine the effects of elevated air speeds on body heat balance and thermal comfort~\cite{ashrae552017}.
The model has also been tested for its relevance under realistic air flows and \ac{v} produced by different types of fans, such as partial immersion in the elevated flows, and flows with different turbulence intensity~\cite{Huang2014}.
This makes the SET model appropriate for predicting the thermophysiological effects of elevated air speed during heatwaves and to determine humidity-dependent temperature thresholds at which electric fans would become detrimental.
We compared our results with those obtained from the \ac{phs} model.
Which allows the analytical evaluation of the thermal stress experienced by a subject working in a hot environment.
We contextualized our results by comparing them to the most extreme weather events recorded in the 115 most populous cities worldwide.
Finally, to help policymakers and people worldwide we chose to incorporate the model into the CBE Thermal Comfort Tool~\cite{Tartarini2020} and the Python package \verb|pythermalcomfort|~\cite{Tartarini2020a}.
The former is an open-source, free-to-use, web-based online tool that provides interactive plots and displays the results to the user (\url{https://comfort.cbe.berkeley.edu}).
The latter is a Python package that allows users to calculate the most common thermal comfort indices in compliance with the main thermal comfort standards.
Our tools should allow users to determine when people can safely use elevated air speeds to cool themselves.