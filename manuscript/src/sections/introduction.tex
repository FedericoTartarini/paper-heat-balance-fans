%! Author = sbbfti
%! Date = 10/06/2020

% Document

%\section*{Questions} todo
%
%\begin{itemize}
%    \item Shall we consider the effect of humidifying the outdoor air to cool it and reduce the thermal strain due to increased sweating?
%    This could be a solution to the problem that sweat rate is limited to 500mL/h
%    \item Which journal shall we target?
%    \item Shall we provide results for an older adult as well?
%    If so how will we calculate the w for the elderly?
%    \item Are we planning to conduct any study in the lab with people, do we have a thermal manikin to estimate latent heat losses?
%    \item Shall we look at the weather model and estimate how the prevalence of hot days will change in the next 50 years or so?
%\end{itemize}

\section{Introduction}\label{sec:introduction}

Anthropogenic activities are the primary causes of global warming.
Since the pre-industrial period human activities are estimated to be the primary cause of a 1~°C increase in the Earth's global average temperature~\cite{GlobalWa91:online}.
However, since 1981 the combined land and ocean temperature has increased at an average rate of 0.18~°C per decade~\cite{GlobalCl28:online}.
As a consequence, nine of the ten warmest years on records have occurred since 2005~\cite{ClimateC26:online}.
It is estimated that by 2030 the rate at which temperature increases is going to accelerate due to the heating imbalance between the greenhouse gases and the oceans' thermal inertia~\cite{ClimateC26:online}.
Consequently, associated risk to rising temperatures such as heatwaves are expected to increase in length, intensity and frequency due to global warming~\cite{Whatharm75:online}.
There is not an universally definition of heatwave however, the \ac{wmo} and \ac{who} describe them as: ``a periods of unusually hot and dry or hot and humid weather that have a subtle onset and cessation, a duration of at least two or three days, usually with a discernible impact on human and natural systems"~\cite{WMO2015}.
Age (very young and elderly), pre-existing conditions, low income, prolonged outdoor activities, and social isolation all play a negative role and increase the heat risk during heat waves~\cite{WMO2015}.
Heatwaves may also put an additional burden on the health, emergency, energy, water and transportation sectors.
For example, during heat waves peak demand consumption increases due to an increase energy demand for cooling, while the efficiency of the grid and the power plant decrease.
This may result in power shortages and blackouts.

Both hot and dry conditions can affect the heat balance of the human body.
Rapid changes in heat gains can compromise the ability of the body to regulate its core temperature and can worsen health of those with pre-existing conditions or result in illnesses which may even be life threatening~\cite{WMO2015}.
The \ac{who}, and the \ac{wmo} together with national governments provide public health guidance for people to minimise the heat during heat stress.
For example, they suggest to keep the body hydrated, to keep out of the heat and keep the home cool.
However, despite the quantitative evidence provided by \mycite{Jay2015} and \mycite{Jay2019a}, the \ac{who} does not recommend the use of electric fan when temperatures exceed 35~°C~\cite{HeatandH28:online, WMO2015}.
% todo add more examples of standards that discourage the se of fans at high temperatures
This suggestion, however, could be very detrimental considering the fact that many people during heat waves may not have the financial means to cool the space using air conditioning, nor to visit public spaces which are air-conditioned.
These people could be in fact the elderly, people with mobility issues or poorer people living either in remote areas or in developing parts of the world.
% todo say something about the fact that many poor people even have limited access to electricity
These groups are already more at risk than the rest of the population, and by discouraging them from utilizing electrical fans when temperatures exceed 35~°C can exacerbate heat stress.
Despite the fact that elevated air movement increases the sensible heat gains for \ac{t-db} higher than \ac{t-sk}, elevated air movement also increases the \ac{e-max}.
Consequently, the use of fans is beneficial even if \ac{t-db} is higher than \ac{t-sk}, as long as the increase in \ac{e-sk} can compensate for the extra sensible heat gains.
%In addition, the critical \ac{t-db} above which the use of fans to increase the air speed becomes detrimental, varies as a function of \ac{rh}.
%With the critical \ac{t-db} decreasing as a function of \ac{rh}.

Electrical fans are relative inexpensive to buy, pedestal fans are available for as low as 20 USD, they are energy efficient, do not have any installation cost and can consume as low as few Watts of power to operate~\cite{Yang2015a, Lipczynska2018a, Jay2019a}.
Electrical fans can therefore be used as an alternative or to complement compressor-based air conditioning~\cite{Yang2015a, Lipczynska2018a, Jay2019a}.
Research has also shown that can they be used in tropical hot and humid climates to increase occupants satisfaction with their thermal environment~\cite{Lipczynska2018a}.

To better understand under which environmental conditions the use of fans would be beneficial, \mycite{Jay2015} provide a model to estimate the combinations of \ac{t-db} and \ac{rh} at which the use of electrical fans would be beneficial.
However, their model had the following limitations, it does not: consider radial asymmetry, estimates iteratively \ac{t-sk} and \ac{t-cr}, allow user to change the value of \ac{met} or \ac{clo}.
Moreover, their results are based on a single air speed velocity 4.5~m/s which cannot be achieved by most of ceiling fans~\cite{Raftery2019},and pedestal fans~\cite{Yang2015a}.
Consequently, to overcome the above mentioned limitations, we are proposing the use of the heat balance model developed by~\mycite{GaggeSET} to estimate heat loss and physiological variables as a function of environmental and personal factors.
Our models, allowed us to estimate the combinations of \ac{t-db}, \ac{rh}, \ac{t-r}, \ac{v}, \ac{clo}, \ac{met} at which the use of elevated air speed would be beneficial during heatwaves.
In addition, to help the overall community we developed an open-source, free to use, web-based online tool which provides interactive plots and displays the results to the user.
Our tool allows building practitioners, educators and policy makers to determine when people can safely use elevated air speeds to cool their body.

%\begin{itemize}
%    \item I do not really see the benefit of the model proposed by Hosper since: 1) they set the limit of sweating to 440 mL/h; and they limit 2) the amount of water than I person can spray on him self to 116 mL/h. This would not affect the results in the SET model since with the SET model the estimated sweating rate is lower.
%    On the other hand it affects the result in the Ollie model since the exponential growth of the sweat rate in their model is more marked.
%    Moreover, very dry climates could benefit from using evaporative cooling.
%\end{itemize}