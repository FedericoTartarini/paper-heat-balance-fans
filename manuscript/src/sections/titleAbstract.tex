%! Author = Federico
%! Date = 10/06/2020

\begin{frontmatter}

\title{Application of Gagge's energy balance model to determine humidity-dependent temperature thresholds for healthy adults using electric fans during heatwaves}

%% Group authors per affiliation:
\author[sinBerBest]{Federico Tartarini\corref{mycorrespondingauthor}}
\ead{federicotartarini@berkeley.edu}
\author[CBE]{Stefano Schiavon}
\author[USYD]{Ollie Jay}
\author[CBE]{Edward Arens}
\author[CBE]{Charlie Huizenga}
\cortext[mycorrespondingauthor]{Corresponding author}

\address[sinBerBest]{SinBerBEST, Berkeley Education Alliance for Research in Singapore, Singapore}
\address[CBE]{Center for the Built Environment, University of California, Berkeley, USA}
 \address[USYD]{Sydney School of Health Sciences, Faculty of Medicine and Health, The University of Sydney, Sydney, Australia}

\begin{abstract}
    The WHO classifies heatwaves as one of the most dangerous natural hazards causing more than 166,000 deaths from 1998--2017.
    Their frequency is steadily increasing and their are becoming more intense.
    Heatwaves mostly affect specific socioeconomic and demographic groups, those who work outdoors, and those with pre-existing health conditions.
    
    Electric fans are an efficient, and sustainable solution to cool people.
    They are, for most applications, the cheapest cooling technology available on the market.
    However, many national and international guidelines actively advise people not to use them when indoor air temperatures exceed the skin temperature, approximately 35~$^{\circ}$C\@.
    
    We used a human energy balance model, to verify the validity of those recommendations and to determine under which environmental (air temperature, humidity, air speed and mean radiant temperature) and personal (metabolic rate, clothing) conditions the use of fans would be beneficial.
    
    We found that current guidelines are too restrictive.
    Electric fans can be used safely and effectively in a wide range of conditions.
    They can be used even if the indoor dry-bulb temperature exceeds the skin temperature since they significantly increase the amount of sweat that evaporates from the skin.
    The use of elevated air speeds (0.8~m/s) increases the critical operative temperature at which heat strain is expected to occur by an average of \var{avg_increase_t_strain_v_08}~$^{\circ}$C for relative humidity values above 22~\%\@.
    We also analysed the most extreme weather events with a 20 year return period recorded in the 115 most populous cities worldwide and we concluded that in 103 of them the use of fans would have been beneficial.
    
    To help researchers, building practitioners, and policymakers better understand under which conditions electric fans can be safely used to cool people, we have also developed a free, open-source, and easy-to-use online tool (\url{https://comfort.cbe.berkeley.edu}).
\end{abstract}

\begin{keyword}
Resilience \sep Heat stress \sep Cooling \sep Air movement \sep Heat Strain \sep Energy \sep Open-source tool
\end{keyword}

\end{frontmatter}
