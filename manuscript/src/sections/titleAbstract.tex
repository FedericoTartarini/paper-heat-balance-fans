%! Author = sbbfti
%! Date = 10/06/2020

\begin{frontmatter}

\title{Energy balance model to determine benefits of using electric fans during heatwaves}

%% Group authors per affiliation:
\author[sinBerBest]{Federico Tartarini\corref{mycorrespondingauthor}}
\author[CBE]{Stefano Schiavon}
\author[USYD]{Ollie Jay}
\cortext[mycorrespondingauthor]{Corresponding author}
\ead{support@elsevier.com}

\address[sinBerBest]{SinBerBEST, Berkeley Education Alliance for Research in Singapore, Singapore}
\address[CBE]{Center for the Built Environment, University of California, Berkeley, USA}
\address[USYD]{Sydney School ofHealth Sciences, Faculty ofMedicine and Health, The University ofSydney, Sydney, Australia}

\begin{abstract}
    The WHO classifies heatwaves as one of the most dangerous natural hazards and it estimates that from 1998--2017 more than 166000 people died worldwide.
    Global warming is worsening the issue.
    While everyone can be experience health related issues during heatwaves, the poor and those people with pre-existing conditions are mostly at risk.
    The former are vulnerable since they may not have the financial means to cool the air indoors.
    Electric fans are an effective and economical solution to provide cool the body.
    Fans are inexpensive to buy and to operate.
    The air movement generated not only increases the convection heat losses from the skin to the environment but also increases the evaporative heat transfer coefficient.
    We used an heat balance model to determine under which environmental (e.g., air temperature, air speed) and personal (i.e., metabolic rate, clothing) conditions the use of fans would be beneficial indoors.
    Results showed that the use of electrical fans would be beneficial up to temperatures as high as 40~°C regardless of the relative humidity.
    In addition, fans would have been beneficial in during all the most severe heat waves recorded in the last decade.
    Our results show that most of the current national and international guidelines underestimate the benefit of elevated air speeds in extreme temperature events.
    We, therefore, created a web-based tool which can be used by practitioners, educators and policy makers to determine under which environmental conditions the use of electrical fans may have a benefit, albeit marginal.
\end{abstract}

\begin{keyword}
Air movement \sep Extreme temperature events \sep Energy conservation \sep Hydration \sep Heat stress
\end{keyword}

\end{frontmatter}
