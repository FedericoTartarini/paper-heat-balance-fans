%! Author = sbbfti
%! Date = 10/06/2020

\begin{frontmatter}

\title{Energy balance model to determine benefits of using electric fans during heatwaves}

%% Group authors per affiliation:
\author[sinBerBest]{Federico Tartarini\corref{mycorrespondingauthor}}
\author[CBE]{Stefano Schiavon}
\author[USYD]{Ollie Jay}
\cortext[mycorrespondingauthor]{Corresponding author}
\ead{support@elsevier.com}

\address[sinBerBest]{SinBerBEST, Berkeley Education Alliance for Research in Singapore, Singapore}
\address[CBE]{Center for the Built Environment, University of California, Berkeley, USA}
\address[USYD]{Sydney School ofHealth Sciences, Faculty ofMedicine and Health, The University ofSydney, Sydney, Australia}

\begin{abstract}
    The WHO classifies heatwaves as one of the most dangerous natural hazards and it estimates that from 1998--2017 more than 166'000 people died worldwide.
    Heatwaves mostly affect specific socioeconomic (e.g., poor) and demographics groups (e.g., elderly), those who work outdoors, and those with pre-existing conditions.
    Electric fans are an effective, efficient and economical solution to cool the human body.
    Fans are one of the cheapest cooling technology available on the market.
    However, many national and international guidelines discourage people from using them when indoor air temperatures exceed 35~$^{\circ}$C\@.
    To verify the validity of those recommendations, we used an heat balance model to determine under which environmental (e.g., air temperature, air speed) and personal (i.e., metabolic rate, clothing) conditions the use of fans would be beneficial for cooling people.
    Our results show that most of the current national and international guidelines underestimate the beneficial effect of elevating air speed in extreme temperature events.
    Electrical fan can safely be used in a wide range of conditions even if the indoor dry-bulb temperature exceeds the skin temperature.
    The use of elevated air speeds would on average increase by 2.0~$^{\circ}$C the critical temperature at which heat strain is expected to occur.
    It should also be noted that the maximum operative temperature at which the use of elevated air speeds becomes detrimental is inversely proportional to the relative humidity.
    To better help users understanding under which environmental and personal conditions electrical fans can effectively cool the human body, we also developed a free, easy-to-use web-based tool.
\end{abstract}

\begin{keyword}
Air movement \sep Extreme temperature events \sep Energy conservation \sep Hydration \sep Heat stress
\end{keyword}

\end{frontmatter}
