%! Author = Federico
%! Date = 10/06/2020

\begin{frontmatter}

\title{A human energy balance model to determine the upper temperature and humidity thresholds for healthy adults using electric fans during heatwaves}

% todo SS possible journals nature climate change or sustainability; Journal of Exposure Science & Environmental Epidemiology

%% Group authors per affiliation:
\author[sinBerBest]{Federico Tartarini\corref{mycorrespondingauthor}}
\ead{federicotartarini@berkeley.edu}
\author[CBE]{Stefano Schiavon}
\author[CBE]{Edward Arens}
\author[CBE]{Charlie Huizenga}
 \author[USYD]{Ollie Jay}
\cortext[mycorrespondingauthor]{Corresponding author}

\address[sinBerBest]{SinBerBEST, Berkeley Education Alliance for Research in Singapore, Singapore}
\address[CBE]{Center for the Built Environment, University of California, Berkeley, USA}
 \address[USYD]{Sydney School of Health Sciences, Faculty of Medicine and Health, The University of Sydney, Sydney, Australia}

\begin{abstract}
    The WHO classifies heatwaves as one of the most dangerous natural hazards causing more than 166,000 deaths from 1998--2017.
    Heatwaves mostly affect specific socioeconomic and demographic groups, those who work outdoors, and those with pre-existing health conditions.
    
    Electric fans are an effective, efficient, and sustainable solution to cool people.
    They are, for most applications, the cheapest cooling technology available on the market.
    However, many national and international guidelines discourage people from using them when indoor air temperatures exceed the skin temperature, approximately 35~$^{\circ}$C\@.
    
    To verify the validity of those recommendations, we extended the applicability of an existing human energy balance model to determine under which environmental (e.g., air temperature, humidity) and personal (metabolic rate, clothing) conditions the use of fans would be beneficial.
    
    We found that current guidelines are too restrictive.
    Electric fans can be used safely and effectively in a wide range of conditions.
    They can be used even if the indoor dry-bulb temperature exceeds the skin temperature since they significantly increase the amount of sweat that evaporates from the skin.
    The use of elevated air speeds increases the critical operative temperature at which heat strain is expected to occur by an average of \var{avg_increase_t_strain_v_08}~$^{\circ}$C for relative humidity values above 22~\%\@.
    The operative temperature above which the use of elevated air speeds becomes detrimental is inversely proportional to relative humidity.
    
    To help researchers, building practitioners, and policymakers to better understand under which conditions electric fans can be safely used to cool people, we have developed a free, open-source, and easy-to-use online tool (\url{https://comfort.cbe.berkeley.edu})
\end{abstract}

\begin{keyword}
Cooling \sep Heatwave \sep Energy conservation \sep Hydration \sep Heat stress \sep Physiology \sep Air movement \sep open-source calculation tool
\end{keyword}

\end{frontmatter}
