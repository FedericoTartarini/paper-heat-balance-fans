%! Author = sbbfti
%! Date = 10/06/2020

\begin{frontmatter}

\title{Energy balance model to determine benefits of using electric fans during heatwaves}

%% Group authors per affiliation:
\author[sinBerBest]{Federico Tartarini\corref{mycorrespondingauthor}}
\author[CBE]{Stefano Schiavon}
\author[USYD]{Ollie Jay}
\cortext[mycorrespondingauthor]{Corresponding author}
\ead{support@elsevier.com}

\address[sinBerBest]{SinBerBEST, Berkeley Education Alliance for Research in Singapore, Singapore}
\address[CBE]{Center for the Built Environment, University of California, Berkeley, USA}
\address[USYD]{Sydney School ofHealth Sciences, Faculty ofMedicine and Health, The University ofSydney, Sydney, Australia}

\begin{abstract}
    The WHO classifies heatwaves as one of the most dangerous natural hazards and it estimates that from 1998--2017 more than 166000 people died worldwide.
    Heatwaves mostly affect the health of those individuals with pre-existing conditions and of those who may not have the financial means to cool their homes.
    Electric fans are an effective and economical solution to cool the body.
    Fans are inexpensive to buy and to operate.
    Air movement significantly increases the latent heat loss from the skin.
    We used an heat balance model to determine under which environmental (e.g., air temperature, air speed) and personal (i.e., metabolic rate, clothing) conditions the use of fans would be beneficial for cooling people.
    Results, showed that the maximum operative temperature at which the use of fans is beneficial in inversely proportional to the  value of the relative humidity.
%    However, regardless of the relative humidity, elevated air speed would be beneficial up to temperatures as high as 40~°C.
%    In addition, fans would have been beneficial in during all the most severe heatwaves recorded in the last decade.
    Our results show that most of the current national and international guidelines underestimate the benefit of elevated air speeds in extreme temperature events.
    We, therefore, created a web-based tool which can be used by practitioners, educators and policy makers to determine under which environmental and personal conditions the use of electrical fans may have a benefit, albeit marginal, to cool the human body.
\end{abstract}

\begin{keyword}
Air movement \sep Extreme temperature events \sep Energy conservation \sep Hydration \sep Heat stress
\end{keyword}

\end{frontmatter}
