%! Author = sbbfti
%! Date = 10/06/2020

\begin{frontmatter}

\title{Human energy balance model to determine upper limit temperature and humidity conditions for the use of electrical fans}

%% Group authors per affiliation:
\author[sinBerBest]{Federico Tartarini\corref{mycorrespondingauthor}}
\ead{federicotartarini@berkeley.edu}
\author[CBE]{Stefano Schiavon}
\author[CBE]{Edward Arens}
\author[CBE]{Charlie Huizenga}
% \author[USYD]{Ollie Jay}
\cortext[mycorrespondingauthor]{Corresponding author}


\address[sinBerBest]{SinBerBEST, Berkeley Education Alliance for Research in Singapore, Singapore}
\address[CBE]{Center for the Built Environment, University of California, Berkeley, USA}
% \address[USYD]{Sydney School of Health Sciences, Faculty of Medicine and Health, The University of Sydney, Sydney, Australia}

\begin{abstract}
    The WHO classifies heatwaves as one of the most dangerous natural hazards and it estimates that from 1998--2017 more than 166'000 people died worldwide.
    Heatwaves mostly affect specific socioeconomic (e.g., poor) and demographics groups (e.g., elderly), those who work outdoors, and those with pre-existing conditions.
    Electric fans are an effective, efficient, and sustainable solution to cool people.
    Fans are one of the cheapest cooling technology available on the market.
    However, many national and international guidelines discourage people from using them when indoor air temperatures exceed 35~$^{\circ}$C\@.
    To verify the validity of those recommendations, we extended the applicability of an existing heat balance model to determine under which environmental (e.g., air temperature, air speed) and personal (i.e., metabolic rate, clothing) conditions the use of fans would be beneficial.
    We found that current guidelines are too restrictive. Electrical fan can safely be used in a wide range of conditions even if the indoor dry-bulb temperature exceeds the skin temperature.
    The use of elevated air speeds would on average increase by 2.0~$^{\circ}$C the critical temperature at which heat stress is expected to occur.
    The maximum operative temperature at which the use of elevated air speeds becomes detrimental is inversely proportional to the relative humidity.
    We developed a free, easy-to-use web-based tool to to determine under which environmental and personal conditions electrical fans can be used to cool people.
\end{abstract}

\begin{keyword}
Cooling \sep Heatwave \sep Energy conservation \sep Hydration \sep Heat stress \sep Physiology
\end{keyword}

\end{frontmatter}
