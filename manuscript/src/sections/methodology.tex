%! Author = sbbfti
%! Date = 10/06/2020

\section{Methodology}\label{sec:methodology}

% todo the set model has the issue with the fact that it iterates over that strange parameter lenght of time = 6o

In this manuscript, we used the heat balance model that \mycite{GaggeSET} developed to derive the \ac{set} to determine when the use of elevated air moment (e.g.. \ac{v} $>$ 0.2 m/s) would be beneficial to cool the human body.
The heat balance model allows us to estimate how environmental (i.e., \ac{t-db}, \ac{t-r}, \ac{v}, \ac{rh}) and personal factors (i.e., \ac{clo}, \ac{met}) influence both latent and sensible components of the \ac{q-sk}, and the \ac{q-res}.
Moreover, it can be used to estimate the value of some physiological variables such as, \ac{t-sk}, and \ac{t-cr}.

Section~\ref{subsec:energy-balance} describe the main Equations used by the model to derive the results.
We have published the algorithm we used to calculate the results in a public repository.
% todo add link to repository
% todo say that Gagge's equation is written in a un-used language, hence we have converted it to Python and JavaScript
Moreover, we added it to the pythermalcomfort Python package~\cite{Tartarini2020a} and the CBE thermal comfort tool~\cite{Tartarini2020}.
The former can be used by Python users to calculate the results presented in this manuscript.
The latter is a web-based tool that can be used to generate interactive figures which show the environmental conditions under which the use of elevated air speeds is beneficial.

\subsection{Energy Balance}\label{subsec:energy-balance}

The human body exchanges both sensible and latent heat with its surrounding environment.
Sensible heat is transferred via conduction, convection and radiation (\acs{c-r} + \acs{c-res}).
While latent heat losses occur from evaporation of sweat  (\acs{e-rsw}), moisture diffused through the skin  (\acs{e-dif}), and respiration (\acs{e-res})
The energy balance in the human body is described by:

\begin{equation}
    M - W = (C + R + E_{sk}) + (C_{res} + E_{res}) + (S_{sk} + S_{cr})\label{eq:heat-balance}
\end{equation}

This equation assumes that the body comprises two main thermal compartments: the skin and the core.
If the exogenous and endogenous heat gains cannot be compensated by the heat losses, then both the \ac{s-sk}, and the \ac{s-cr} increase and in turn the \ac{t-sk}, and \ac{t-cr} rise, respectively.

The amount of sensible heat gains or losses from the human body to its environment can be expressed as a function of environmental, and personal factors.
The former are \ac{t-db}, \ac{t-r}, \ac{v}, and \ac{rh}.
While the latter are \ac{met}, and \ac{clo}~\cite{ASHRA2017}.

The equations used to determine sensible and latent heat loss are based on fundamental heat transfer theory, while the coefficient were estimated empirically~\cite{ASHRA2017}.

\subsubsection{Body Surface Area, (\acs{body-a})}

All the terms presented in Equation~\ref{eq:heat-balance} are reported in power per unit of human \ac{body-a}.
Equation~\ref{eq:dubois} can be used to estimate \ac{body-a} as a function of the \ac{body-w} and \ac{body-h} of the person~\cite{DuBois}.

\begin{equation}
    A_{body} = 0.202 m^{0.425} l^{0.725}\label{eq:dubois}
\end{equation}

% look at this source

\subsubsection{Sensible Heat Loss from Skin, (\acs{c-r})}

Sensible heat losses from the human body mainly occur from convection and radiation from the skin to the environment.
The total amount of \ac{c-r} can be described as a function of the \ac{t-sk}, \ac{t-op}, \ac{r-cl}, \ac{f-cl}, and \ac{h}.
The equation can be expressed as:

\begin{equation}
    C+R=\frac{t_{s k}-t_{o}}{R_{c l}+1 /\left(f_{c l} h\right)}\label{eq:c-r}
\end{equation}

\begin{equation}
    f_{cl}=1.0 + 0.31 I_{cl} \label{eq:f-cl}
\end{equation}

\begin{equation}
    h=h_{c} + h_{r} = \max(3, 8.6 v^{0.53}) p_{atm}^{0.53} + 4 \varepsilon \sigma \frac{A_{\mathrm{r}}}{A_{body}}\left[273.2+\frac{\left(t_{\mathrm{cl}}+\overline{t_{r}}\right)}{2}\right]^{3}\label{eq:h}
\end{equation}

% todo describe variables that did not appear in the text before

Where \ac{t-op} varies as a function of \ac{h-c}, \ac{h-r}, \ac{t-r} and \ac{t-db}, and it is described by:

\begin{equation}
    t_{o}=\frac{h_{r} \bar{t}_{r}+h_{c} t_{db}}{h_{r}+h_{c}}\label{eq:t-op}
\end{equation}

The \ac{t-sk} is calculated iteratively by the heat balance model since it varies as a function of the heat loss from the human body towards its environment and the heat transferred from the core to the skin node.
\Ac{t-cl} can be calculated as a function of the \ac{t-op}, \ac{t-sk}, \ac{r-cl} and the resistance of the air layer.

\subsubsection{Latent Heat Loss from Skin, (\acs{e-sk})}

The \acf{e-sk} comprises two terms the \ac{e-rsw} and the \ac{e-dif}.
\ac{e-sk} depends on the \ac{w}, \ac{p-sk} normally assumed to be that of saturated water vapor at \ac{t-sk}, \ac{p-a}, \ac{f-cl}, \ac{h-e}, and \ac{r-e-cl}.

\begin{equation}
    E_{s k}=E_{rsw}+E_{dif}=\frac{w\left(p_{s k, s}-p_{a}\right)}{R_{e, c l}+1 /\left(f_{c l} h_{e}\right)}\label{eq:latent-skin}
\end{equation}

Despite the fact that Equation~\ref{eq:latent-skin} is expressed as a function of \ac{w}.
The human body does not regulates \ac{w} directly but, rather, it regulates the sweat rate~\cite{ASHRA2017}.
Skin wettedness varies as a function of the activity of the sweat glands and the environmental conditions~\cite{ASHRA2017}.
While, theoretically \ac{w} can range from 0 to 1, in practice, \ac{w} is strongly correlated with thermal stress and warm discomfort, consequently there is a \ac{w-max} for sustained activity for healthy and acclimatized humans~\cite{ASHRA2017}.

We estimated the \ac{w-max} value and the \ac{m-sweat} using the \mycite{GaggeSET} model.

\subsubsection{Respiratory Losses, (\acs{q-res})}
The human body exchanges both sensible and latent heat with its environment.
The \acf{q-res} equals the sum of the \ac{c-res} and the \ac{e-res}.
The value of \ac{q-res} is can be determined using the following simplified equation~\cite{ASHRA2017}:

\begin{equation}
    q_{res} = C_{res} + E_{res} = 0.0014M(34-t_{a}) + 0.0173M(5.87-p_{a})\label{eq:respiratory-losses}
\end{equation}

\subsection{Data Analysis}\label{subsec:data-analysis}

The heat balance model was used to estimate the sensible and latent heat loss and physiological parameters (e.g., \ac{m-sweat}, \ac{t-cr}).
We calculated the results for \ac{t-op} ranging from 28 and 55~$^{\circ}$C at 0.5~$^{\circ}$C intervals, \ac{rh} ranging from 0 to 100~\% at 5~\% intervals and for the discrete values of \ac{v} = 0.2, 0.8 and 4.5~m/s.
In this paper we will be referring to `still air' condition when air velocities are below \ac{v}~=~0.2 m/s.
This definition is in accordance with the ASHRAE 55--2017 Standard~\cite{ashrae552017} and allowed us to compare our results with~\mycite{Jay2015}.
We assumed \ac{t-r} to be equal to \ac{t-db}, \ac{clo}~=~0.5 clo, and \ac{met}~=~1.0 met, unless otherwise stated.
Results for different combinations of environmental and personal conditions can be generated using our online tool.
We reported heat loss per unit of surface area (i.e., \ac{e-sk}).
To obtain the absolute value of heat loss or gain the results can be multiplied by \ac{body-a}.
Thermal stress was assumed to occur when \ac{w} (which only depends on the value of \ac{v}) reaches its maximum value \ac{w-max}.
We assumed that the use of electrical fans is detrimental when the value of \ac{t-cr} calculated for values of \ac{v} higher than 0.2~m/s exceeds the value determined for the `still air' condition.

\subsection{Heatwave weather data}

% todo descibe weather data used

\subsection{Elderly}

% todo say something how we calculated w for the elderly

\subsection{Open-source tools}

% todo descibe the CBE tool and pythermalcomfort equation to generate the figures