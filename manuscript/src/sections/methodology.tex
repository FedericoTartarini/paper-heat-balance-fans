%! Author = sbbfti
%! Date = 10/06/2020

\section{Methodology}\label{sec:methodology}

We used the heat balance model that \mycite{GaggeSET} originally developed to derive the \ac{set} to determine when the use of elevated air moment (i.e., \ac{v} $>$ 0.2 m/s) would be beneficial to cool people.
The heat balance model allows to estimate how environmental (i.e., \ac{t-db}, \ac{t-r}, \ac{v}, \ac{rh}) and personal factors (i.e., \ac{clo}, \ac{met}) influence both latent and sensible components of the \ac{q-sk}, and the \ac{q-res}.
Moreover, it can be used to estimate the value of some physiological variables such as \ac{t-sk}, and \ac{t-cr}.
It should be noted that \ac{t-db} and \ac{v} are the average values measured at different heights over a period of at least three minutes.
These heights are 0.1, 0.6 and 1.1~m for seated people and 0.1, 1.1 and 1.7~m for standing occupants~\cite{ashrae552017}.

Section~\ref{subsec:energy-balance} describe the main Equations used by the model to derive the results.

\subsection{Energy Balance}\label{subsec:energy-balance}

The human body exchanges both sensible and latent heat with its surrounding environment.
Sensible heat is transferred via conduction, convection and radiation (\acs{c-r} + \acs{c-res}).
While latent heat loss occurs from the evaporation of sweat (\acs{e-rsw}), moisture diffused through the skin  (\acs{e-dif}), and respiration (\acs{e-res})
The energy balance in the human body is described by:

\begin{equation}
    M - W = C + R + E_{sk} + C_{res} + E_{res} + S_{sk} + S_{cr}\label{eq:heat-balance}
\end{equation}

This equation assumes that the body comprises two main thermal compartments: the skin and the core.
If the exogenous and endogenous heat gains cannot be compensated by heat loss, then both the \ac{s-sk}, and the \ac{s-cr} increase and in turn the \ac{t-sk}, and \ac{t-cr} rise, respectively.
One of the main differences between our proposed model and the one used by \mycite{Jay2015} is that their model does not accounts for heat stored in the core or the skin compartment, hence, they assume the values of \ac{t-sk} and \ac{t-cr} to be constant.
Calculating how \ac{t-sk} and \ac{t-cr} vary as a function of different environmental and personal factors allowed us to better predict how much heat the body exchanges with its surrounding environment.

The amount of sensible heat gains or losses from the human body to its environment can be expressed as a function of environmental (\ac{t-db}, \ac{t-r}, \ac{v}, and \ac{rh}), and personal factors (\ac{met}, and \ac{clo})~\cite{ASHRA2017}.

The equations used to determine sensible and latent heat loss are based on fundamental heat transfer theory, while the coefficient were estimated empirically~\cite{ASHRA2017}.

\subsubsection{Body Surface Area}

All the terms presented in Equation~\ref{eq:heat-balance} are reported in power per unit of human \ac{body-a}.
Equation~\ref{eq:dubois} can be used to estimate \ac{body-a} as a function of the \ac{body-w} and \ac{body-h}~\cite{DuBois}.

\begin{equation}
    A_{body} = 0.202 m^{0.425} l^{0.725}\label{eq:dubois}
\end{equation}

In thermal comfort research this value is generally assumed to be constant and equal to 1.8 m$^{2}$.
\mycite{Jay2015} also consider this value to be constant, however, it should be noted that this in an approximation.
Several other equations have been developed to estimate \ac{body-a} and in the function of the \verb|pythermalcomfort| tool we allow users to specify \ac{body-a} as input value.
Alternative equations to the DuBois equation are available.
A review paper concluded that most of the proposed equations in the literature were in agreement with each other to estimate \ac{body-a} for adults with a healthy weight and standard physique~\cite{Redlarski2016}.

\subsubsection{Sensible Heat Loss from Skin}

Sensible heat loss from the human body mainly occur from convection and radiation from the skin to the environment.
The total amount of \ac{c-r} can be described as a function of \ac{t-sk}, \ac{t-op}, \ac{r-cl}, \ac{f-cl}, and \ac{h}.
The equation can be expressed as:

\begin{equation}
    C+R=\frac{t_{s k}-t_{o}}{R_{c l}+1 /\left(f_{c l} h\right)}\label{eq:c-r}
\end{equation}

\begin{equation}
    f_{cl}=1.0 + 0.31 I_{cl} \label{eq:f-cl}
\end{equation}

\begin{equation}
    h=h_{c} + h_{r} = \max(3, 8.6 v^{0.53}) p_{atm}^{0.53} + 4 \varepsilon \sigma \frac{A_{\mathrm{r}}}{A_{body}}\left[273.2+\frac{\left(t_{\mathrm{cl}}+\overline{t_{r}}\right)}{2}\right]^{3}\label{eq:h}
\end{equation}

Where the ratio between \ac{a-r} and \ac{body-a} is assumed to be 0.70 for a sitting person and 0.73 for a standing person~\cite{Fanger1967}.
The \ac{e} is close to unity (typically 0.95) and the \ac{sigma} is a constant.
The value of \ac{t-op} varies as a function of the \ac{h-c}, \ac{h-r}, \ac{t-r} and \ac{t-db}, and it is described by:

\begin{equation}
    t_{o}=\frac{h_{r} \bar{t}_{r}+h_{c} t_{db}}{h_{r}+h_{c}}\label{eq:t-op}
\end{equation}

In our model, the values of \ac{t-sk} is calculated iteratively since it varies as a function of the heat loss from the human body towards its environment and the heat transferred from the core to the skin node, as shown in Source Code~\ref{lst:pythonCode}.
\Ac{t-cl} can be calculated as a function of \ac{t-op}, \ac{t-sk}, \ac{r-cl} and the resistance of the air layer.

\subsubsection{Latent Heat Loss from Skin, (\acs{e-sk})}

The \acf{e-sk} comprises two terms the \ac{e-rsw} and the \ac{e-dif}.
\ac{e-sk} depends on the \ac{w}, \ac{p-sk} normally assumed to be that of saturated water vapor at \ac{t-sk}, \ac{p-a}, \ac{f-cl}, \ac{h-e}, and \ac{r-e-cl}.

\begin{equation}
    E_{s k}=E_{rsw}+E_{dif}=\frac{w\left(p_{s k, s}-p_{a}\right)}{R_{e, c l}+1 /\left(f_{c l} h_{e}\right)}\label{eq:latent-skin}
\end{equation}

Despite the fact that Equation~\ref{eq:latent-skin} is expressed as a function of \ac{w}.
The human body does not regulates \ac{w} directly but, rather, it regulates the sweat rate~\cite{ASHRA2017}.
Skin wettedness varies as a function of the activity of the sweat glands and the environmental conditions~\cite{ASHRA2017}.
While, theoretically \ac{w} can range from 0 to 1, in practice, \ac{w} is strongly correlated with thermal stress and warm discomfort, consequently there is a \ac{w-max} for sustained activity for healthy and acclimatized humans~\cite{ASHRA2017}.

We estimated \ac{w-max} using the following equation:

\begin{equation}
    w_{max}=
\begin{cases}
    0.38 V^{-0.29} & \text{if } I_{cl} = 0 \text{ (i.e., naked)} \\
    0.59 V^{-0.08} & \text{if } I_{cl} > 0
\end{cases}
\end{equation}

On the other hand, \mycite{Jay2015} adjusted the value of \ac{w-max} based on fan use and age.
For young adults, they assumed \ac{w-max} to be equal to 0.65 for the `fan on' condition and 0.85 for the `fan off' condition.
% todo add references
These values are higher than those estimated by the \mycite{GaggeSET} model.

\subsubsection{Respiratory Losses, (\acs{q-res})}
The human body exchanges both sensible and latent heat with its environment.
The \acf{q-res} equals the sum of the \ac{c-res} and the \ac{e-res}.
The value of \ac{q-res} is can be determined using the following simplified equation~\cite{ASHRA2017}:

\begin{equation}
    q_{res} = C_{res} + E_{res} = 0.0014M(34-t_{a}) + 0.0173M(5.87-p_{a})\label{eq:respiratory-losses}
\end{equation}

\subsection{Data Analysis}\label{subsec:data-analysis}

The heat balance model was used to estimate sensible and latent heat loss and several physiological parameters (e.g., \ac{m-sweat}, \ac{t-cr}).
We calculated the results for \ac{t-op} ranging from 28 to 55~$^{\circ}$C at 0.5~$^{\circ}$C intervals, \ac{rh} ranging from 0 to 100~\% at 5~\% intervals and for the discrete values of \ac{v} = 0.2, 0.8 and 4.5~m/s.
In this paper we will be referring to `still air' condition when air velocities are below \ac{v}~=~0.2 m/s.
This definition is in accordance with the ASHRAE 55--2017 Standard~\cite{ashrae552017} and allowed us to compare our results with those obtained by \mycite{Jay2015}.
We assumed \ac{t-r} to be equal to \ac{t-db}, \ac{clo}~=~0.5~clo, and \ac{met}~=~1.1~met, unless otherwise specified.
It could be argued that some people during heatwaves may be wearing less clothes than that, hence, a value of \ac{clo} equal to 0.36 clo (i.e., walking shorts, short-sleeve shirt and sandals) would be more appropriate, however, we wanted to use a more conservative value.
Results for different combinations of environmental and personal conditions can be generated using our online tool.
In this manuscript we assumed the \ac{i-cl} to be constant and equal to 1 and 0.45, as assumed by \mycite{GaggeSET}, for naked and clothed people, respectively.
User can, however, change this value in the source code.
We reported heat losses per unit of surface area.
Thermal stress was assumed to occur when either of the following parameters reached its maximum value \ac{w}, skin blood flow or \ac{m-sweat}.
The former assumption was based on the fact that there is a \ac{w-max} for sustained activity for healthy and acclimatized humans~\cite{ASHRA2017}.
The other two assumptions are based on the fact that \mycite{GaggeSET} state that serious danger of fatality exists when blood flow from core to skin is maximal or sweating reaches its maximum.
We assumed that the use of electrical fans is detrimental when the value of \ac{t-cr} calculated for values of \ac{v} higher than 0.2~m/s exceeds the value determined for the `still air' condition.

Results were calculated using the  \verb|pythermalcomfort| Python package~\cite{Tartarini2020a} function \verb|use_fans_heatwaves|.
A copy of the algorithm we used to calculate the results can also be found in \ref{sec:python_code}.
Lines in Figures~\ref{fig:comparison_models}, \ref{fig:results_model_2}, \ref{fig:comparison_air_speed}, \ref{fig:met_clo}, \ref{fig:energy_storage_delta}, and \ref{fig:use_fans_and_population} were smoothed using the \verb|Scipy| function \verb|ndimage.gaussian_filter1d|.
Moreover, we develop a tool that can be used to generate interactive figures that show the environmental conditions under which the use of elevated air speeds is beneficial.
This tool was added and integrated in the CBE Thermal Comfort Tool~\cite{Tartarini2020}.

\subsection{Weather Data}\label{subsec:weather-data}

To better understand in which locations worldwide the use of electrical fans would be beneficial. 
We compared the results obtained with the heat balance model with the climatic data provided in the 2017 ASHRAE Handbook--Fundamentals~\cite{ASHRA2017} and the records from the Emergency Events Database (EM-DAT) which contains a list of the deadliest heatwaves recorded from 1936 to the present date~\cite{EMDATThe70:online}.

From the ASHRAE climatic design dataset we extracted information regarding the maximum extreme \ac{t-db} and \ac{t-wb} recorded across more than 5000 stations worldwide with a 10 year return period.
For more information about the ASHRAE climate design dataset please refer to Chapter 14 of the 2017 ASHRAE Handbook--Fundamentals~\cite{ASHRA2017}.
Location of the stations and their respective maximum extreme dry-bulb temperatures are shown in Figure~\ref{fig:world-map}.
We did not show data from stations with a maximum temperature lower than 26~$^\circ$C\@ since we are only interested in assessing the benefit of using fans during hot days.

\begin{figure}[thb!]
    \centering
    \includegraphics[width=\textwidth]{figures/world-map.png}
    \caption{Shows the location of each weather station that was included in the analysis and the maximum extreme dry-bulb temperature with a 10 year return period.}
    \label{fig:world-map}
\end{figure}

Few data was available for the Sub-Saharan Africa where approximately 40~\% of the poorest people in the world reside and where climate change may be an acute threat~\cite{PovertyO1:online}.

For each location we assumed that both the maximum extreme \ac{t-db} and wet-bulb temperatures could be recorded at the same time.
This is an approximation and it is arguably overestimating the most extreme condition due to the fact that the likelihood that both conditions would occur at the same time is extremely low.
However, we assumed this to be the most extreme heat event that could occur in each location.
In addition, we assumed that during heatwaves \ac{t-db} and \ac{rh} indoors would be equal to \ac{t-db} and \ac{rh} outdoors.
Conditions indoors may, however, be slightly less severe than outdoors since the thermal mass of the building may dump and shift peaks in outdoor temperature.
At the same time, the opposite scenario can also occur if there is a significant amount of internal load or solar gains indoors.

The EM-DAT contains detailed information on when the heatwave occurred, the location, the number of deaths, and the maximum temperature recorded.
However, it does not contain information about the \ac{rh} which is essential in determining whether the use of electrical fans would have been beneficial or not.

\subsection{City Population Data}\label{subsec:population-data}

We obtained the city population data from the demographic statistics database which is compiled and maintained by the \ac{un} statistic division~\cite{UNdatare88:online}.
The database was last updated in August 2020 and we used it to gather information about the number of people who live in the 115 most populous cities in the world.
When available we used the population of the urban agglomerate rather than of the city administrative boundary.
We then combined the population with the ASHRAE weather data to determine during extreme heat events: i) how many people were at high risk of experiencing \ac{t-db} higher than 35~$^{\circ}$C\@; and ii) how many people would benefit from the use of electrical fans.
As previously mentioned in Section~\ref{subsec:weather-data} weather data were not available for all the major cities in the world.
Consequently, we had to exclude the following cities from our analysis: Lagos in Nigeria, Dar es-Salaam and Mwanza in Tanzania, Dhaka in Bangladesh, Faisalabad in Pakistan, Zibo and Zhongshan in China, Addis Ababa in Ethiopia, and Bandung in Indonesia.
A full list of the cities we included in the analysis is provided in the \ref{sec:pop_weather}.

\subsection{Elderly}\label{subsec:elderly}

% todo say something how we calculated w_crit for the elderly