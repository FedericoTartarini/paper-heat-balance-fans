%! Author = sbbfti
%! Date = 10/06/2020

\section{Methodology}\label{sec:methodology}

We used the heat balance model that Gagge et al. (\citeyear{GaggeSET}) developed to derive the \ac{set}.
This model allowed us to estimate the \ac{q-sk}, and \ac{q-res} as a function of the following input variables: \ac{t-db}, \ac{t-r}, \ac{rh}, \ac{v}, \ac{clo}, and \ac{met}.

\subsection{Energy Balance}\label{subsec:energy-balance}

The human body exchanges both sensible and latent heat with its surrounding environment.
Sensible heat flow comprises a combination of conduction, convection and radiation (\acs{c-r} + \acs{c-res}).
While latent heat losses occur from evaporation of sweat  (\acs{e-rsw}), moisture diffused through the skin  (\acs{e-dif}), and moisture during respiration (\acs{e-res})
The energy balance in the human body is summarized in Eq.~\ref{eq:heat-balance}.

\begin{equation}
    M - W = (C + R + E_{sk}) + (C_{res} + E_{res}) + (S_{sk} + S_{cr})\label{eq:heat-balance}
\end{equation}

This equation assumes that the body comprises two thermal compartments: the skin and the core.
In steady state condition (i.e., the \ac{t-sk}, and \ac{t-cr} remain constant) the \ac{s-cr}, and the \ac{s-sk} are equal to zero.
The amount of sensible heat gains or losses from the human body to its environment can be expressed as a function of environmental, and personal factors.
The former are \ac{t-db}, \ac{t-db}, \ac{v}, and \ac{rh}.
While the independent personal factors are \ac{met}, and \ac{clo}~\cite{ASHRA2017}.

\subsubsection{Thermal Exchanges with Environment}\label{subsubsec:thermal-exchanges-with-environment}

This section summarizes the main equations used by the \ac{set} model to calculate the various terms presented in Eq.~\ref{eq:heat-balance}.
The equations to estimate sensible and latent heat loss are based on fundamental heat transfer theory, while the coefficient used were estimated empirically~\cite{ASHRA2017}.

\paragraph{Body Surface Area}

All the terms presented in Eq.~\ref{eq:heat-balance} are reported in power per unit of human \ac{body-a}.
DuBois et al.~(\citeyear{DuBois}) Equation~\ref{eq:dubois} can be used to estimate \ac{body-a} as a function of the \ac{body-w} and \ac{body-h} of the person.

\begin{equation}
    A_{body} = 0.202 m^{0.425} l^{0.725}\label{eq:dubois}
\end{equation}

% look at this source

\paragraph{Sensible Heat Loss from Skin}

Sensible heat losses from the human body mainly occur from convection and radiation from the skin to the environment.
The total amount of \ac{c-r} can be described as a function of the \ac{t-sk}, \ac{t-op}, \ac{r-cl}, \ac{f-cl}, and \ac{h}.
The equation can be expressed as:

\begin{equation}
    C+R=\frac{t_{s k}-t_{o}}{R_{c l}+1 /\left(f_{c l} h\right)}\label{eq:c-r}
\end{equation}

\begin{equation}
    f_{cl}=1.0 + 0.31 R_{cl} / 0.155\label{eq:f-cl}
\end{equation}

\begin{equation}
    h=h_{c} + h_{r} = \max(3, 8.6 v^{0.53}) p_{atm}^{0.53} + 4 \varepsilon \sigma \frac{A_{\mathrm{r}}}{A_{body}}\left[273.2+\frac{\left(t_{\mathrm{cl}}+\overline{t_{r}}\right)}{2}\right]^{3}\label{eq:h}
\end{equation}

% todo say that skin temperature is calc iteratively
% todo describe variables that did not appear in the text before

Where \ac{t-op} varies as a function of \ac{h-c}, \ac{h-r}, \ac{t-r} and \ac{t-sk}, and it is described by:

\begin{equation}
    t_{o}=\frac{h_{r} \bar{t}_{r}+h_{c} t_{db}}{h_{r}+h_{c}}\label{eq:t-op}
\end{equation}

\paragraph{Latent Heat Loss from Skin}

\begin{equation}
    E_{s k}=E_{rsw}+E_{dif}=\frac{w\left(p_{s k, s}-p_{a}\right)}{R_{e, c l}+1 /\left(f_{c l} h_{e}\right)}\label{eq:latent-skin}
\end{equation}

Where the \ac{w} can vary from 0 and 1.

% todo describe variables that did not appear in the text before

Despite the fact that Eq.~\ref{eq:latent-skin} is expressed as a function of \ac{w}.
The human body does not regulates the \acl{w} but, rather, it regulates the sweat rate~\cite{ASHRA2017}.
Skin wettedness varies as a function of the activity of the sweat glands and the environmental conditions~\cite{ASHRA2017}.
While theoretically skin wettedness can range from 0 to 1.
In practice, \ac{w} is strongly correlated with thermal stress and warm discomfort, consequently there is a practical upper limit for sustained activity for healthy and acclimatized humans~\cite{ASHRA2017}.
The \ac{set} models can be used to calculate the value of \ac{w-max}.

% todo write equation to estimate sweat losses

\paragraph{Respiratory Losses}

\begin{equation}
    C_{res} + E_{res} = 0.0014M(34-t_{a}) + 0.0173M(5.87-p_{a})\label{eq:respiratory-losses}
\end{equation}

\subsubsection{Data Analysis}

% todo descibe the data analsysis