\documentclass[11pt,a4paper,roman]{moderncv}        % possible options include font size ('10pt', '11pt' and '12pt'), paper size ('a4paper', 'letterpaper', 'a5paper', 'legalpaper', 'executivepaper' and 'landscape') and font family ('sans' and 'roman')
% \usepackage[spanish,es-lcroman]{babel}

% moderncv themes
\moderncvstyle{classic}                            % style options are 'casual' (default), 'classic', 'oldstyle' and 'banking'
\moderncvcolor{black}                              % color options 'blue' (default), 'orange', 'green', 'red', 'purple', 'grey' and 'black'
% \renewcommand{\familydefault}{\sfdefault}         % to set the default font; use '\sfdefault' for the default sans serif font, '\rmdefault' for the default roman one, or any tex font name
\nopagenumbers{}                                  % uncomment to suppress automatic page numbering for CVs longer than one page

% character encoding
\usepackage[utf8]{inputenc}                       % if you are not using xelatex ou lualatex, replace by the encoding you are using

\renewcommand\labelitemi{•}

% adjust the page margins
\usepackage[scale=0.8]{geometry}
\usepackage{enumitem}
\setlist[enumerate]{align=left}
%\setlength{\hintscolumnwidth}{3cm}                % if you want to change the width of the column with the dates
%\setlength{\makecvtitlenamewidth}{10cm}           % for the 'classic' style, if you want to force the width allocated to your name and avoid line breaks. be careful though, the length is normally calculated to avoid any overlap with your personal info; use this at your own typographical risks...

% personal data
\name{}{Dr. Federico Tartarini}
% \title{Postdoctoral scholar}                               % optional, remove / comment the line if not wanted
\address{Postdoctoral scholar at SinBerBEST, Berkeley Education Alliance for Research in Singapore}{1 Create Way, \#11-01, CREATE Tower}{Singapore 138602} % optional, remove / comment the line if not wanted; the "postcode city" and and "country" arguments can be omitted or provided empty
% \phone[mobile]{000-000-000-000}                   % optional, remove / comment the line if not wanted
% \phone[fixed]{+2~(345)~678~901}                    % optional, remove / comment the line if not wanted
% \phone[fax]{+3~(456)~789~012}                      % optional, remove / comment the line if not wanted
\email{federicotartarini@berkeley.edu}                               % optional, remove / comment the line if not wanted
%\homepage{www.johndoe.com}                         % optional, remove / comment the line if not wanted
% \extrainfo{Berkeley Education Alliance for Research in Singapore}                 % optional, remove / comment the line if not wanted
%\photo[64pt][0.4pt]{picture}                       % optional, remove / comment the line if not wanted; '64pt' is the height the picture must be resized to, 0.4pt is the thickness of the frame around it (put it to 0pt for no frame) and 'picture' is the name of the picture file
%\quote{Some quote}                                 % optional, remove / comment the line if not wanted

% to show numerical labels in the bibliography (default is to show no labels); only useful if you make citations in your resume
%\makeatletter
%\renewcommand*{\bibliographyitemlabel}{\@biblabel{\arabic{enumiv}}}
%\makeatother
%\renewcommand*{\bibliographyitemlabel}{[\arabic{enumiv}]}% CONSIDER REPLACING THE ABOVE BY THIS

% bibliography with mutiple entries
%\usepackage{multibib}
%\newcites{book,misc}{{Books},{Others}}
%----------------------------------------------------------------------------------
%            content
%----------------------------------------------------------------------------------
\begin{document}
%-----       letter       ---------------------------------------------------------
% recipient data
\recipient{Cover letter to the Editor}{of the journal of Building and Environment}
\date{\today}
\opening{Dear Professor Nesreen K. Ghaddar,}

\makelettertitle

We submit an original research article entitled ``Application of Gagge's energy balance model to determine humidity-dependent temperature thresholds for healthy adults using electric fans during heatwaves'' for consideration to be published in Building and Environment.
The authors are Federico Tartarini, Stefano Schiavon, Ollie Jay, Edward Arens, and Charlie Huizenga.
We are experts in thermal comfort, heat stress, and thermal physiology.
We confirm that this manuscript it is not currently under consideration for publication in any other journal.

Several current health guidelines worldwide underestimate the benefit that air movement has in cooling people and discourage them from using fans when air temperatures exceed 35~$^{\circ}$C.
This significantly reduces the applicability of this affordable, sustainable and effective cooling strategy during heatwaves.
We used a validated energy balance model, developed by Gagge et al. (1971), to determine how environmental (air speed and mean radiant temperature) and personal (metabolic rate, clothing) conditions affect humidity-dependent temperature thresholds at which fans can be used.

We compare our results with those obtained by laboratory-based physiological studies and other biophysical models developed by Jay et al. (2015), Morris et al. (2021), and the Predicted Heat Strain (PHS) model present in the ISO 7933 standard. Moreover, to better understand in which locations and how many people worldwide would benefit from using electric fans, we compare our results with climate and population data from ASHRAE and the United Nations population database.

Electric fans are a cheaper and far more energy-efficient alternative to compressor-based air conditioning and we conclude that:
\begin{enumerate}[itemindent=.5cm,nolistsep]
    \item Health guidelines regarding electric fan use during heatwaves currently underestimate the benefit of air movement.
    \item Electric fans can be used safely even when the indoor dry-bulb temperature exceeds the skin temperature, across a wide range of relative humidity.
    \item Over the last 20 years, even during the most extreme weather events, the use of fans would have been beneficial in 103 of the 115 most populous cities worldwide.
\end{enumerate}

To help people worldwide to determine under which environmental and personal conditions the use of fans is beneficial, we include the model used in the manuscript embodied in two free and open-source tools.

Please do not hesitate to contact me if I can be of any further assistance. Thank you in advance for considering our research for publication in Building and Environment.

\bigskip

Dr. Federico Tartarini, PhD\\
Postdoctoral scholar at SinBerBEST, BEARS


\end{document}